\documentclass{amsart}
\usepackage{a4wide}
\usepackage{verbatim}
\usepackage{hyperref}

\setlength{\topmargin}{-1cm}
\setlength{\oddsidemargin}{-1cm}
\setlength{\evensidemargin}{-1cm}
\setlength{\textwidth}{18cm}
\setlength{\textheight}{24.5cm}

\newcommand{\NOTEhelp}{1.1}
\newcommand{\NOTEproblems}{1.2}
\newcommand{\NOTEsemi}{1.3}
\newcommand{\NOTEcolon}{1.4}
\newcommand{\NOTEpercent}{1.5}
\newcommand{\NOTElasteval}{1.6}
\newcommand{\NOTEcolonequals}{1.7}
\newcommand{\NOTEfuncdef}{1.8}
\newcommand{\NOTErestart}{1.9}
\newcommand{\NOTEunassign}{1.10}
\newcommand{\NOTEevalf}{2.1}
\newcommand{\NOTEevalfdigits}{2.2}
\newcommand{\NOTEdigits}{2.3}
\newcommand{\NOTEvariables}{3.1}
\newcommand{\NOTEcase}{3.2}
\newcommand{\NOTElongvars}{3.3}
\newcommand{\NOTEreserved}{3.4}
\newcommand{\NOTEgreek}{3.5}
\newcommand{\NOTEsubscripts}{3.6}
\newcommand{\NOTEstar}{4.1}
\newcommand{\NOTEmissingstar}{4.2}
\newcommand{\NOTEtwoletters}{4.3}
\newcommand{\NOTEtimesbrackets}{4.4}
\newcommand{\NOTEtimesminus}{4.5}
\newcommand{\NOTEpowers}{4.6}
\newcommand{\NOTEpowerbrackets}{4.7}
\newcommand{\NOTEpowerminus}{4.8}
\newcommand{\NOTEsqrt}{4.9}
\newcommand{\NOTEsubs}{5.1}
\newcommand{\NOTEsolvesubs}{5.2}
\newcommand{\NOTEexpand}{5.3}
\newcommand{\NOTEsimplify}{5.4}
\newcommand{\NOTEsymbolic}{5.5}
\newcommand{\NOTEtestequal}{5.6}
\newcommand{\NOTEsort}{5.7}
\newcommand{\NOTEcollect}{5.8}
\newcommand{\NOTEcoeff}{5.9}
\newcommand{\NOTEpi}{6.1}
\newcommand{\NOTEbadpi}{6.2}
\newcommand{\NOTEexp}{6.3}
\newcommand{\NOTEimaginary}{6.4}
\newcommand{\NOTEsolve}{7.1}
\newcommand{\NOTEroots}{7.2}
\newcommand{\NOTEmultisolve}{7.3}
\newcommand{\NOTEnosols}{7.4}
\newcommand{\NOTEenvexplicit}{7.5}
\newcommand{\NOTEallsolutions}{7.6}
\newcommand{\NOTEfsolve}{7.7}
\newcommand{\NOTEfsolveinitial}{7.8}
\newcommand{\NOTEfroots}{7.9}
\newcommand{\NOTEidentity}{7.10}
\newcommand{\NOTEstandard}{8.1}
\newcommand{\NOTEfuncbrackets}{8.2}
\newcommand{\NOTEabs}{8.3}
\newcommand{\NOTElog}{8.4}
\newcommand{\NOTElogbase}{8.5}
\newcommand{\NOTEexpx}{8.6}
\newcommand{\NOTEhyp}{8.7}
\newcommand{\NOTEsinsquared}{8.8}
\newcommand{\NOTEcosec}{8.9}
\newcommand{\NOTEarctan}{8.10}
\newcommand{\NOTEpossqrt}{8.11}
\newcommand{\NOTEarrow}{9.1}
\newcommand{\NOTEbadfuncdef}{9.2}
\newcommand{\NOTEseriesdef}{9.3}
\newcommand{\NOTEbadderiv}{9.4}
\newcommand{\NOTEmultifunc}{9.5}
\newcommand{\NOTEshowfuncdef}{9.6}
\newcommand{\NOTEbasicplot}{10.1}
\newcommand{\NOTEvertrange}{10.2}
\newcommand{\NOTEscaling}{10.3}
\newcommand{\NOTEplotpredef}{10.4}
\newcommand{\NOTEtwoplots}{10.5}
\newcommand{\NOTEdiscont}{10.6}
\newcommand{\NOTEnumpoints}{10.7}
\newcommand{\NOTEparametric}{10.8}
\newcommand{\NOTEparametricbrackets}{10.9}
\newcommand{\NOTEparametricrange}{10.10}
\newcommand{\NOTEimplicitplot}{10.11}
\newcommand{\NOTEgrid}{10.12}
\newcommand{\NOTEsaveplot}{10.13}
\newcommand{\NOTEdisplay}{10.14}
\newcommand{\NOTElistplot}{10.15}
\newcommand{\NOTElistplotstyle}{10.16}
\newcommand{\NOTElistplotgen}{10.17}
\newcommand{\NOTEwithplots}{10.18}
\newcommand{\NOTEreloadplots}{10.19}
\newcommand{\NOTEdiff}{11.1}
\newcommand{\NOTEdiffvar}{11.2}
\newcommand{\NOTEmultidiff}{11.3}
\newcommand{\NOTEleibniz}{11.4}
\newcommand{\NOTEimplicitdiff}{11.5}
\newcommand{\NOTEinertdiff}{11.6}
\newcommand{\NOTEtaylor}{11.7}
\newcommand{\NOTEtaylorconv}{11.8}
\newcommand{\NOTEint}{12.1}
\newcommand{\NOTEintconst}{12.2}
\newcommand{\NOTEdefint}{12.3}
\newcommand{\NOTEimproper}{12.4}
\newcommand{\NOTEinertint}{12.5}
\newcommand{\NOTEseq}{13.1}
\newcommand{\NOTEsum}{13.2}
\newcommand{\NOTEadd}{13.3}
\newcommand{\NOTErememberstar}{14.1}
\newcommand{\NOTEnostar}{14.2}
\newcommand{\NOTEfracbrac}{14.3}
\newcommand{\NOTEexpbrac}{14.4}
\newcommand{\NOTEshape}{14.5}
\newcommand{\NOTEremovedefs}{14.6}
\newcommand{\NOTEbadplotvar}{14.7}
\newcommand{\NOTEbadexp}{14.8}
\newcommand{\NOTEbadpiagain}{14.9}


\newwrite\keyfile
\immediate\openout\keyfile=notekeys.tex

\newcommand{\seenote}[1]{\textbf{[#1]}}

\begin{document}
\begin{center}
 {\Large Maths with Maple --- Maple reference}
\end{center}
\vspace{4ex}

\newcounter{notecounter}
\newcommand{\note}[1]{%
 \stepcounter{notecounter}
 \immediate\write\keyfile{\string\newcommand{\string\NOTE#1}{\arabic{section}.\arabic{notecounter}}}
 \item[(\arabic{section}.\arabic{notecounter})]
}

\section{Generalities}
\setcounter{notecounter}{0}

\begin{itemize}
 \note{help} For help, type \verb~?~ and press
 ENTER.  (Alternatively, you can click where it says
 ``Help'' at the top right of the Maple window.)  For help
 on plotting (for example) you can type \verb~?plot~ or
 \verb~help(plot);~ and press ENTER.  Alternatively, you can
 click on ``Help'', then select ``Topic search'' or ``Full
 text search'' from the resulting menu, then enter
 ``plotting'' in the search box. 
%
 \note{problems} See Section~\ref{sec-problems}
 for advice on some common problems. 
% (I will update this section of the online
%  version of these notes, as and when people report more
%  problems to me.) 
%
 \note{semi} Commands must end with a semicolon
 (or occasionally a colon).  If you see the message
 ``Warning, premature end of input'', then you have probably
 left out the semicolon. 
%
 \note{colon}
 If you use a colon then Maple will carry out your command
 but will not tell you the answer; this is sometimes useful
 when calculating intermediate results that would take a lot
 of space to print.  
%
 \note{percent} You can use the symbol \verb~%~
 to refer to the last thing that Maple calculated, and
 \verb~%%~ to refer to the one before that, and so on. 
 Even if you use a colon to prevent Maple from displaying a
 result, you can still use \verb~%~ to refer to that result. 
%
 \note{lasteval}
 If you move around a worksheet inserting things in
 different places, you should remember that \verb~%~ refers
 to the \textbf{most recent} result, which may not be the
 one that you see directly above the line where you are
 typing.  
%
 \note{colonequals}
 It is important to understand the difference between
 \verb~=~ and \verb~:=~.  If we enter $x:=3;$, this tells
 Maple that $x$ is definitely equal to $3$, so Maple will
 replace $x$ by $3$ whenever it sees an $x$.  If instead we
 enter $x=3$, we are mentioning the question of whether $x$
 might be $3$, or temporarily considering the case when
 $x=3$.  This has no effect in itself, until it is used in a
 command like \verb~solve(x=3)~ or \verb~eval(x^2+5,x=3)~
 or \verb~limit((x^2-9)/(x-3),x=3)~. 
%
 \note{funcdef}
 If you want to define a function (say $f(x)=x^2+3$) it does
 \emph{not} work to enter \verb~f(x):=x^2+3;~.  See
 Section~\ref{sec-funcdef}. 
%
 \note{restart}
  Enter \verb~restart;~ to restart Maple, removing any values
  that you may have assigned to variables.  You should
  generally do this at the beginning of each new exercise,
  otherwise stray values left over from previous exercises
  can cause trouble and confusion.  You can also restart by
  clicking the button with a circulating arrow at the
  right-hand end of the Maple toolbar. 
%
 \note{unassign}
  If you just want to clear a single variable (say $x$) you
  can enter \verb~unassign('x')~ instead.  Note the single
  quote marks around the $x$. 
\end{itemize}

\section{Numerical approximation}
\setcounter{notecounter}{0}

\begin{itemize}
 \note{evalf} To get an explicit numerical approximation for an
  expression, use the \verb~evalf()~ function.  For example,
  \verb~evalf(Pi)~ gives $3.141592654$, and
  \verb~evalf(x/3)~ gives $0.3333333333 x$. 
%
 \note{evalfdigits} The \verb~evalf()~ function generally gives an
  answer to $10$ significant figures.  If you want to
  calculate $\pi^2/6$ to $50$ digits (for example), you can
  enter \verb~evalf[50](Pi^2/6)~.  Similarly,
  \verb~evalf[12](1/7)~ gives $1/7$ to 12 decimal places. 
%
 \note{digits} As an alternative, you can enter
   \verb~Digits:=100;~ to tell Maple to use $100$ digits in
   all calculations until further notice.  This approach is
   usually a better choice, as it ensures that the extra
   digits will be kept in subsequent calculations.  You may
   assume that \verb~Digits:=100~ will give you exactly
   $100$ significant figures, but the real truth is a little
   more complicated than that. 
\end{itemize}

\section{Notation for variables}
\setcounter{notecounter}{0}

\begin{itemize}
 \note{variables} Most variables will have single-letter
  names, such as $x$ or $A$. 
%
  \note{case} Note that case is significant: Maple regards
  $A$ as different from $a$, and $x$ as different from $X$. 
  Similarly, $\sin(x)$ must be entered as \verb~sin(x)~ and
  not \verb~Sin(x)~; this is a fairly common source of
  trouble. 
%
  \note{longvars} It is perfectly permissible to use long
  names for variables, such as \verb~eqns~ or \verb~sols~ or
  \verb~SpeedOfLight~. 
%
  \note{reserved} However, you may not use words that
  already have a special meaning in Maple.  In particular,
  you cannot call a variable \verb~roots~ or \verb~name~,
  for example. 
%
  \note{greek} Greek letters can be entered using
  their english names.  For example, if you enter
  \verb~theta+phi~ then Maple will print it as
  $\theta+\phi$.  However, the constant
  $\pi\simeq 3.14159$ must be entered as \verb~Pi~ (with a
  capital P) not \verb~pi~ (see \NOTEbadpi). 
%
  \note{subscripts} Sometimes one uses subscripted
  variables, such as $x_3$ or $a_{2,4}$ or
  $\text{Solution}_5$.  These should be entered as
  \verb~x[3]~ or \verb~a[2,4]~ or \verb~Solution[5]~. 
\end{itemize}

\section{Algebraic operations}
\setcounter{notecounter}{0}

\begin{itemize}
 \note{star} Use a \verb~*~ for multiplication; for example,
 $2xy$ should be entered as \verb~2*x*y~, and $(2x+1)\sin(3x)$
 should be entered as \verb~(2*x+1)*sin(3*x)~. 
%
 \note{missingstar}
 An error message saying ``Error, missing operator or `;`''
 often indicates a missing \verb~*~.  Things like
 \verb~2(x+y)~ (with a number followed by an open bracket)
 are particularly dangerous; Maple will (for technical
 reasons that make some sense in the right context) silently
 convert \verb~2(x+y)~ to $2$ without any error message. 
%
 \note{twoletters}
 Note also that Maple will treat \verb~xy~ as a two-letter
 name for a single variable; if you want $x$ times $y$, you
 must enter \verb~x*y~. 
%
 \note{timesbrackets}
 Just as in standard mathematical notation, bracketing is
 important.  To multiply $x+2$ by $x-2$, you must enter
 \verb~(x+2)*(x-2)~, not \verb~x+2*x-2~.  
%
 \note{timesminus}
 Additionally, if you want to multiply
 by something that starts with a minus sign, you must
 enclose it in brackets; for example, enter
 \verb~(x+y)*(-u)~, not \verb~(x+y)*-u~. 
%
 \note{powers} Enter powers using the \verb~^~ character,
 for example \verb~x^6~ for $x^6$. 
%
 \note{powerbrackets}
 Just as in standard mathematical notation, bracketing is
 important.  To raise $x+2$ to the power $n+5$, you must
 enter \verb~(x+2)^(n+5)~, not \verb~x+2^n+5~.  To get the
 cube root of $x$, enter \verb~x^(1/3)~, not
 \verb~x^1/3~.  
%
 \note{powerminus}
 Additionally, if you want to raise something to a power
 that starts with a minus sign, you must enclose it in
 brackets; for example, enter \verb~(x+y)^(-2)~, not
 \verb~(x+y)^-2~. 
%
 \note{sqrt} The square root of $x$ can be entered as
 \verb~sqrt(x)~ or \verb~x^(1/2)~.  Similarly,
 $\sqrt{b^2-4ac}$ is \verb~sqrt(b^2-4*a*c)~.  For more
 general roots you should use power notation, such as
 \verb~(x+y)^(1/4)~ for the fourth root of $x+y$. 
\end{itemize}

\section{Algebraic manipulation}
\setcounter{notecounter}{0}

Suppose we have a complicated expression called $A$, which we want to
simplify or otherwise manipulate. 
\begin{itemize}
 \note{subs} To substitute for some or all of the variables in
  $A$, use the \verb~subs~ command.  For example, if $A$ is
  $(a+b)^n$, you can change the $b$ to $1$ and the $n$ to
  $p/q$ by entering \verb~subs(b=1,n=p/q,A)~, which gives
  $(a+1)^{p/q}$.  The syntax \verb~subs({b=1,n=p/q},A)~
  will also work. 
%
 \note{solvesubs}
  Often the equations to be used in \verb~subs(...)~ will
  come from the \verb~solve(...)~ command.  For example, to
  find the value of $x^2+y^2$ at the point where $x+y=6$ and
  $x-y=2$, enter \verb~sol:=solve({x+y=6,x-y=2},{x,y})~ and
  then \verb~subs(sol,x^2+y^2)~. 
%
 \note{expand} Enter \verb~expand(A)~ to expand everything out.  For
  example, enter \verb~expand((x+y)^3)~ to expand
  $(x+y)^3$, giving $x^3+3x^2y+3xy^2+y^3$. 
%
 \note{simplify} To simplify $A$, try entering
  \verb~simplify(A)~.  If this 
  does not do what you want, you can try \verb~factor(A)~. 
  If $A$ involves trigonometric functions, try
  \verb~simplify(expand(convert(A,exp)))~. 
%
 \note{symbolic} If Maple is refusing to simplify $(a^4)^{1/4}$ to $a$ (for
  example), you can try \verb~simplify(A,symbolic)~.  You should be
  aware that the answer \textbf{may not be correct} in all cases,
  because $(a^4)^{1/4}$ really is different from $a$ when
  $a<0$.  Similarly, the simplification $\ln(e^x)=x$ is not
  always valid when $x$ is a complex number, so
  \verb~simplify(ln(exp(x)))~ just gives $\ln(e^x)$, but
  \verb~simplify(ln(exp(x)),symbolic)~ gives $x$. 
%
 \note{testequal} If you suspect that $A$ is equal to some other expression (say
  $B$) and want to check this, the best way is to enter
  \verb~simplify(A-B)~ and see if you get zero. 
%
 \note{sort} If Maple writes an expression with the terms in an unnatural
  order (eg $t^3+t^4+t^2+t$ instead of $t^4+t^3+t^2+t$) then use the
  \verb~sort~ command to rearrange them. 
%
 \note{collect} To collect together all the terms in $A$ involving the same
  power of $x$, enter \verb~collect(A,x)~.  For example, this converts
  $1+ax+bx+cx^2+dx^2$ to $1+(a+b)x+(c+d)x^2$. 
%
 \note{coeff}
  If you just want the
  coefficient of $x^2$ then you can instead enter \verb~coeff(A,x,2)~
  or \verb~coeff(A,x^2)~.  To get the constant term, you
  must enter \verb~coeff(A,x,0)~ --- the syntax
  \verb~coeff(A,x^0)~ will not work. 
\end{itemize}

\section{Constants}
\setcounter{notecounter}{0}


\begin{itemize}
%
 \note{pi} The constant $\pi\simeq 3.1415926536$ should be entered as
  \verb~Pi~, with a capital P.  Maple will generally leave this as a
  symbol; if you want the numerical value, you should use the
  \verb~evalf()~ function. 
%
 \note{badpi}
  If you enter \verb~pi~ with a small p, then Maple will accept it
  without a syntax error and will print it as $\pi$, but will not
  treat it as mathematically equivalent to \verb~Pi~.  If you have an
  expression like $\sin(\pi)$ and Maple stubbornly refuses to convert
  it to zero, then this may be the cause; you may have entered
  \verb~pi~ instead of \verb~Pi~.  You can cure this by entering
  \verb~pi:=Pi;~, which will tell Maple to treat \verb~pi~ the same
  as \verb~Pi~. 
%
 \note{exp} The constant $e\simeq 2.718281828$ \textbf{cannot} (by default)
  be entered as \verb~e~.  Instead, enter \verb~exp(1)~ for $e$, and
  \verb~exp(x)~ for $e^x$.  If you want Maple to treat \verb~e~ as
  meaning \verb~exp(1)~, you can enter \verb~e:=exp(1);~. 
%
 \note{imaginary} The square root of minus one can be
  entered as \verb~I~ (not \verb~i~ or \verb~j~).  
\end{itemize}

\section{Solving}
\setcounter{notecounter}{0}


\begin{itemize}
%
 \note{solve} To solve the equation $x^2-5x+6=0$ (for example), enter
  \verb~solve(x^2-5*x+6=0,{x})~.  Similarly, if you have
  already defined a function $g$, you can enter
  \verb~solve(g(t)=0,{t})~ to solve the equation $g(t)=0$. 
%
  \note{roots}
   The above syntax, with curly brackets around
   the \verb~x~, gives the answer in the form
   $\{x=2\},\{x=3\}$.  If you prefer to have the answer in
   the form $2,3$ (with no ``$x=$'') , you should instead enter
   \verb~solve({x^2-5*x+6=0},x)~ (without curly brackets
   around the \verb~x~). 
%
 \note{multisolve} To solve several equations in several variables, put the
  equations in one set of curly brackets, and the variables in another
  set.  For example, to solve $2x+3y=5$ and $3x+4y=7$ for $x$ and $y$,
  enter 
\begin{verbatim}
 solve({2*x+3*y=5,3*x+4*y=7},{x,y});
\end{verbatim}
%
 \note{nosols} If there are no solutions, Maple will just give you a new
  prompt without printing any answer at all. 
%
 \note{envexplicit} If the answer involves the word
  \verb~RootOf~, it may help to enter
  \verb~_EnvExplicit:=true;~ and try again.  
%
 \note{allsolutions} If the problem involves trigonometric
  functions, then there will typically be infinitely many
  solutions, but Maple will only list a few of them.  To get
  all solutions, enter \verb~_EnvAllSolutions:=true;~. 
%
 \note{fsolve} To get approximate numerical solutions, use \verb~fsolve~
  instead of \verb~solve~.  This will generally only find one solution
  (unless the equation to be solved is very simple).  
%
 \note{fsolveinitial}
  To find more solutions, you must tell \verb~fsolve~ where
  to start looking.  For example,
  \verb~fsolve({sin(x)=0},{x=3})~ finds a root of $\sin(x)$
  near $x=3$.  
%
 \note{froots} As with the \verb~solve()~ command, if you are
  solving for a single variable, you can get the answer in
  two different forms.  For example, \verb~fsolve({x^2=2},x)~
  gives $-1.414213562, 1.414213562$, whereas
  \verb~fsolve({x^2=2},{x})~ gives $\{x=-1.414213562\},
  \{x=1.414213562\}$. 
% 
 \note{identity} Sometimes we need to solve problems like
  this: find $a$ and $b$ such that $a(x-1)^2+b(x+1)^2=x$
  \emph{for all x}.  To tell Maple that this is meant to be
  an identity valid for all $x$, we use the syntax \\
  \verb~solve(identity(a*(x-1)^2+b*(x+1)^2=x,x),{a,b})~. 
  Similarly, to find $u$ such that $sin(t+u)=cos(t)$ for all
  $t$, enter
  \verb~solve(identity(sin(t+u)=cos(t),t),{u})~. 
\end{itemize}

\section{Standard functions}
\setcounter{notecounter}{0}


\begin{itemize}
%
 \note{standard} Most functions can be entered using their
  usual names, for example \verb~sin(x)~ or \verb~ln(y)~. 
%
 \note{funcbrackets}
  However, you must always use brackets: enter
  \verb~tan(x)~ instead of \verb~tan x~, and \verb~sin(2*x)~
  instead of \verb~sin 2x~. 
%
 \note{abs} The absolute value of $x$, traditionally written as
  $|x|$, must be entered as \verb~abs(x)~. 
%
 \note{log} Both \verb~ln(x)~ and \verb~log(x)~ refer to the natural
  logarithm (to base $e$) of $x$. 
%
 \note{logbase}
  If you want to use the logarithm to base $10$ then you
  should enter \verb~log[10](x)~.  In many cases 
  you will actually need to enter \verb~simplify(log[10](x))~ to get
  a useful answer.  Similarly, enter \verb~log[2](x)~ for
  $\log_2(x)$ (the logarithm to base $2$) and so on. 
%
 \note{expx} The function $e^x$ should be entered as \verb~exp(x)~. 
%
 \note{hyp} The hyperbolic function $\sinh(x)=(e^x-e^{-x})/2$ can be
  entered as \verb~sinh(x)~, and similarly for the functions
  $\cosh(x)=(e^x+e^{-x})/2$ and $\tanh(x)=\sinh(x)/\cosh(x)$.  Maple
  is not as good as it should be at dealing with these functions; it
  often works better to write expressions explicitly in terms of $e^x$
  instead. 
%
 \note{sinsquared} The traditional notation $\sin^2(x)$
  refers to the square of $\sin(x)$, which could also be
  written $(\sin(x))^2$ or $\sin(x)^2$.  In Maple we suggest
  that you enter this as \verb~sin(x)^2~.  It will 
  not work to enter \verb~sin^2(x)~.  Similar comments apply to
  $\tan^2(x)$ and so on.  Note that $\sin(x^2)$ means something
  different again: it is the sin of the square of $x$, not the square
  of the sin. 
%
 \note{cosec} In traditional notation $1/\sin(x)$ is sometimes written as
  $\text{csc}(x)$ or $\text{cosec}(x)$.  In Maple \verb~csc(x)~ will
  work, but \verb~cosec(x)~ will not.  Of course, you can also just
  enter \verb~1/sin(x)~. 
%
 \note{arctan} In traditional notation $\tan^{-1}(x)$ or $\arctan(x)$ refers
  to the angle $\theta$ such that $\tan(\theta)=x$.  This is
  completely different from the number $\tan(x)^{-1}=1/\tan(x)$.  In
  Maple you should enter \verb~arctan(x)~ (\textbf{not}
  \verb~tan^(-1)(x)~ or anything like that).  Similar comments apply
  to $\arcsin(x)$, $\text{arctanh}(x)$ and so on. 
%
 \note{possqrt} When $x\geq 0$, the notation \verb~sqrt(x)~ means, by
  definition, the positive square root of $x$.  (There is more to say
  if $x$ is negative or complex, but we pass over that here.)  Thus,
  for example, \verb~sqrt(9)~ is definitely $3$ and not $-3$.  
\end{itemize}

\section{Defining your own functions}\label{sec-funcdef}
\setcounter{notecounter}{0}


\begin{itemize}
%
 \note{arrow} If you want to define $f(x)=x^2$, enter
   \verb~f := (x) -> x^2;~  Note that the arrow symbol is
   typed as a dash (\verb~-~) followed by a greater-than
   sign (\verb~>~), and is not related to the arrow keys
   used to move your cursor.  
%
 \note{badfuncdef} It \textbf{does not} work very well to enter
   \verb~f(x):=x^2;~ instead.  If you do this, then Maple
   will convert $f(x)$ to $x^2$ (when the argument is just
   the letter $x$) but it will not convert $f(y)$ to $y^2$,
   or $f(3)$ to $9$. 
%
 \note{seriesdef}
  Similarly, to define the sequence $a(n)=(-1)^n/n$, enter
  \verb~a := (n) -> (-1)^n/n;~, not \verb~a(n):=(-1)^n/n;~. 
%
 \note{badderiv} It \textbf{does not} work to use notation like
   \verb~f'(x)~ for the derivative of $f(x)$; see
   Section~\ref{sec-diff} instead. 
%
 \note{multifunc}
  You can define a function of several variables in a similar
  way.  For example, to define $g(a,u,v)=u^a+v^a$, enter
  \verb~g := (a,u,v)->u^a+v^a~. 
%
 \note{showfuncdef}
  Suppose you define $g$ as above, and later on you want a
  reminder of the definition.  You should enter
  \verb~print(g)~, and Maple will print out out
  $(a,u,v)\rightarrow u^a+v^a$.  For reasons that we will not
  explore here, it does not work to enter \verb~g~ instead
  of \verb~print(g)~. 
\end{itemize}

\section{Plotting}
\setcounter{notecounter}{0}

\begin{itemize}
%
 \note{basicplot} To plot the graph $y=x^3-x$ from $x=-2$ to $x=2$ (for
  example), enter \verb~plot(x^3-x,x=-2..2);~.  Note that
  there is no ``$y=$'' in the plot command.  Similarly, to
  plot $t/(e^t-1)$ from $t=-1$ to $t=3$, enter
  \verb~plot(t/(exp(t)-1),t=-1..3);~. 
%
 \note{vertrange} To make the vertical axis run from $0$ to $10$ (for
  example), enter \verb~plot(x^3-x,x=-2..2,0..10);~.  Note
  that you must enter \verb~0..10~, \textbf{not}
  \verb~y=0..10~ here.  The horizontal range
  (\verb~x=-2..2~) includes the name of the variable, but
  the vertical range does not. 
%
 \note{scaling} To make the vertical scale the same as the horizontal
  scale, add the option \verb~scaling=constrained~ (eg
  \verb~plot(x^3-x,x=-2..2,scaling=constrained);~). 
%
 \note{plotpredef} If the function to be plotted is
  complicated, it may be convenient to give the definition
  as a separate command, for example
  \verb~y:=t^2+t^3+t^5+t^7+t^11+t^13;~ and then
  \verb~plot(y,t=0..1);~ rather than just\\
  \verb~plot(t^2+t^3+t^5+t^7+t^11+t^13,t=0..1)~. 
%
 \note{twoplots} To plot the graph $y=x^3+x$ in the same picture, enter
  \verb~plot([x^3-x,x^3+x],x=-2..2);~.  (Note that the outer
  pair of brackets are round, and the inner pair are
  square.  Note also the different placing of brackets
  between here and \seenote{\NOTEparametric} below.)  Similarly,
  to plot $\cos(t)$ and $\sin(t)$ 
  together from $t=-3\pi$ to $t=3\pi$, enter
  \verb~plot([sin(t),cos(t)],t=-3*Pi..3*Pi)~. 
%
 \note{discont} To find and skip over discontinuities, add the option
  \verb~discont=true~; for example:
\begin{verbatim}
 plot(frac(x),x=-3..3,discont=true);
\end{verbatim}
%
 \note{numpoints} To improve the accuracy of a graph, use the option
  \verb~numpoints=1000~, for example\\
  \verb~plot(x*sin(Pi/x),x=-1..1,numpoints=1000);~.  (The
  number $1000$ should always give a good picture, but may
  slow things down.  Any number greater than $50$ will
  improve the default picture.)  
%
 \note{parametric} To plot the curve given parametrically by
  $x=1/(1+t^2)$ and $y=\sin(t)$ from $t=-5$ to $t=5$, enter
\begin{verbatim}
 plot([1/(1+t^2),sin(t),t=-5..5]);
\end{verbatim}
  Similarly, to plot the curve $(x,y)=(\sin(t),\cos(2t))$
  for $t=0$ to $2\pi$, enter
  \verb~plot([sin(t),cos(2*t),t=0..2*Pi])~. 
  The option \verb~scaling=constrained~ is often useful
  in this kind of plot.   
%
 \note{parametricbrackets}
  Note carefully the placing of the square brackets above.  If
  you enter \verb~plot([1/(1+t^2),sin(t)],t=-5..5);~
  instead, you get the graphs $y=1/(1+t^2)$ and $y=\sin(t)$
  drawn together in the same picture, which is something
  rather different. 
%
 \note{parametricrange}
  In a parametric plot, if you want the $x$-axis to run from $-3$
  to $3$ and  the $y$-axis from $-2$ to $2$, you should enter \\
  \verb~plot([1/(1+t^2),sin(t),t=-5..5],view=[-3..3,-2..2]);~. 
%
 \note{implicitplot} To plot a curve given implicitly, say
  by the equation $x+y+x^2y^2=1$, enter 
\begin{verbatim}
 plots[implicitplot](x+y+x^2*y^2=1,x=-3..3,y=-3..3);
\end{verbatim}
  Similarly, you can plot the circle $x^2+y^2=4$ like this:
\begin{verbatim}
 plots[implicitplot](x^2+y^2=4,x=-2..2,y=-2..2);
\end{verbatim}
  The option \verb~scaling=constrained~ is often useful
  in this kind of plot. 
%
 \note{grid}
  Implicit plots will rarely give a good picture unless you
  tell Maple to start with a finer grid, which you can do like this:
\begin{verbatim}
 plots[implicitplot](x+y+x^2*y^2=1,x=-3..3,y=-3..3,grid=[200,200]);
\end{verbatim}
  The number 200 generally seems to work well, but you could
  try a larger or smaller one. 
%
 \note{saveplot} To generate and save a plot for future use, enter
  something like this: \\
  \verb~MyPicture:=plot(x^2+x,x=2..4):~.  Note that this 
  ends with a colon, to prevent Maple from printing out vast
  lists of coordinates.  You can then display the picture by
  entering \verb~MyPicture;~ (with a semicolon).  
%
 \note{display}
  You can display two pictures together by entering \\
  \verb~plots[display](MyPicture,MyOtherPicture);~ (for 
  example).  This can also be done without saving the
  pictures separately.  For example, you can plot $y=x^3$
  together with $x^2+y^2=1$ like this:\\
  \verb~plots[display](plot(x^3,x=-1..1),plots[implicitplot](x^2+y^2=1,x=-1..1,y=-1..1));~
%
 \note{listplot}
  You can use \verb~plots[listplot](...)~ to plot a list of
  points.  For example \verb~plots[listplot]([8,7,5,1])~
  plots four points with $y$ coordinates $8$, $7$, $5$ and
  $1$, and $x$ coordinates $1$, $2$, $3$ and $4$. 
  Similarly, \\
  \verb~plots[listplot]([9,9,9,9,9,9,9])~ plots
  seven points, all at height $9$, at $x$-coordinates
  $1,\dotsc,7$. 
%
 \note{listplotstyle}
  By default, \verb~listplot~ draws a line joining the
  specified points.  If you just want the points themselves,
  use the option \verb~style=POINT~, for example 
  \verb~plots[listplot]([8,7,5,1],style=POINT)~
%
 \note{listplotgen}
  If you want, you can specify the $x$-coordinates as well,
  rather than just taking them to be $1,2,3,\dotsc$.  For
  example, \\
  \verb~plots[listplot]([[1,1],[1,-1],[-1,1],[-1,-1]])~
  plots the four points with coordinates $(1,1)$, $(1,-1)$,
  $(-1,1)$ and $(-1,-1)$, which are the four corners of a
  square. 
%
 \note{withplots}
  If you enter \verb~with(plots):~, then Maple will let you
  enter \verb~implicitplot(...)~ instead of
  \verb~plots[implicitplot](...)~, \verb~listplot(...)~
  instead of \verb~plots[listplot](...)~  and
  \verb~display(...)~ instead of \verb~plots[display](...)~,
  and so on.  The same applies to various other commands
  that we have not yet discussed. 
%
 \note{reloadplots}
  If you restart Maple for any reason, you will
  have to enter \verb~with(plots):~ again afterwards. 
\end{itemize}

\section{Differentiation}\label{sec-diff}
\setcounter{notecounter}{0}


\begin{itemize}
%
 \note{diff} To differentiate $x+\sin(x)$ with respect to $x$ (for
  example), enter \verb~diff(x+sin(x),x)~.  This is
  equivalent to $\frac{d}{dx}(x+\sin(x))$ in traditional
  notation.  Similarly, to differentiate $1/(1-t^2)$ with
  respect to $t$ we enter \verb~diff(1/(1-t^2),t)~.  
%
 \note{diffvar} Note that it is necessary to specify the
  variable with respect to which you want to differentiate. 
  If you enter \verb~diff(x^2,x)~ you get $2x$, but if you
  just enter \verb~diff(x^2)~, you get an error message. 
%
 \note{multidiff} To differentiate $y$ three times with respect to $x$
  (for example), enter \verb~diff(y,x,x,x)~ or
  \verb~diff(y,x$3)~.  This is equivalent to $d^3y/dx^3$ in
  traditional notation.  Similarly, to find the second
  derivative of $\sin(\theta)$ with respect to $\theta$ we
  enter \verb~diff(sin(theta),theta$2)~. 
%
 \note{leibniz} In a different kind of traditional notation, you
  could define $f(t)=t^3+4$ (for example), and then $f'(t)$
  would mean the derivative $\frac{d}{dt}f(t)=3t^2$.  The
  Maple equivalent would be to enter \verb~f:=(t)->t^3+4;~
  for the definition, and \verb~D(f)(t)~ for the derivative. 
  Note that \verb~D(f)(t)~ means exactly the same as
  \verb~diff(f(t),t)~, but it is sometimes more convenient. 
%
 \note{implicitdiff} If $x$ and $y$ are related implicitly
  by an equation (say $x+y+x^2y^2=1$) then you can find
  $dy/dx$ in terms of $x$ and $y$ by implicit
  differentiation, like this: 
\begin{verbatim}
 implicitdiff(x+y+x^2*y^2=1,y,x)
\end{verbatim}
%
 \note{inertdiff} Sometimes you just want to mention the derivative
  without actually working it out.  You can do this by
  writing \verb~Diff~ instead of \verb~diff~.  For example,
  if you enter \verb~Diff(x^3+4,x)~ then Maple will print
  out $\frac{d}{dx}(x^3+4)$; if you then enter
  \verb~value(%)~, Maple will convert this to $3x^2$. 
  Similarly, \verb~Diff(t^10,t,t)~ prints as
  $\frac{d^2}{dt^2}t^{10}$, whereas
  \verb~value(Diff(t^10,t,t))~ or \verb~diff(t^10,t,t)~
  gives $90 t^8$. 
%
 \note{taylor} To find the Taylor series of a function $f(x)$ near a
  point $x=a$ to order $n$, enter
  \verb~series(f(x),x=a,n)~.  Note that ``order $n$'' means
  that the highest term allowed is a multiple of $x^{n-1}$,
  and all terms $x^n$ and higher are discarded.  For
  example, \verb~series(1/x,x=2,7)~ gives the Taylor series
  for $1/x$ near $x=2$ including terms up to $(x-2)^6$. 
%
 \note{taylorconv}
  The \verb~series~ command gives an answer including a term
  like $O(x^7)$,  to indicate that there are
   infinitely many more terms, starting with a multiple of
   $x^{12}$.  Maple will not let us do very much with this
   answer until we have converted it to an ordinary
   polynomial (without the $O(x^7)$ term on the end).  To do
   this, use the command \verb~convert(...,polynom)~. 
\end{itemize}

\section{Integration}\label{sec-int}
\setcounter{notecounter}{0}


\begin{itemize}
%
 \note{int} To calculate an indefinite integral like
  $\int\tan(x)\,dx$, enter \verb~int(tan(x),x)~. 
%
 \note{intconst} Maple's answer will never include an arbitrary
  constant '$+c$'.  In some contexts (particularly, solution
  of differential equations) the constant makes a
  difference and must be included.  In other contexts it
  is irrelevant; Maple cannot help you to decide about
  this. 
%
 \note{defint} To calculate a definite integral like
  $\int_1^5t^3\,dt$, enter \verb~int(t^3,t=1..5)~. 
  Similarly, to calculate $\int_0^2\sqrt{1+x}\,dx$, enter
  \verb~int(sqrt(1+x),x=0..2)~. 
%
 \note{improper} It is also possible to have infinite limits.  For
  example, $\int_{-\infty}^\infty (1+x^4)^{-1}dx$ can be 
  entered as 
\begin{verbatim}
 int((1+x^4)^(-1),x=-infinity..infinity)
\end{verbatim}
%
 \note{inertint} As with differentiation, it is sometimes useful to
  mention an integral without evaluating it.  You can do
  this by writing \verb~Int~ instead of \verb~int~.  For
  example, if you enter \verb~Int(x^2,x)~ then Maple will
  print it as $\int x^2\,dx$; if you then enter
  \verb~value(%)~, then Maple will convert it to $x^3/3$. 
\end{itemize}

\section{Sequences and sums}
\setcounter{notecounter}{0}


\begin{itemize}
%
 \note{seq} To generate a sequence of terms, use the \verb~seq()~
  command.  For example, the sequence $1/2,1/4,1/6,1/8,1/10$
  is the sequence of terms $1/(2k)$ for $k=1,2,3,4,5$; it can
  be generated by the command \verb~seq(1/(2*k),k=1..5)~. 
%
 \note{sum} To find the sum of a sequence of terms, use the
  \verb~sum()~ command.  For example,
  $\tfrac{1}{2}+\tfrac{1}{4}+\tfrac{1}{6}+\tfrac{1}{8}+\tfrac{1}{10}$  
  is the sum of the terms $1/(2k)$ for $k$ from $1$ to $5$,
  or in other words $\sum_{k=1}^51/(2k)$.  This can be
  entered as \verb~sum(1/(2*k),k=1..5)~. 
%
 \note{add} Occasionally it works better to use the \verb~add()~
  command instead of \verb~sum()~.  We will not explore the
  difference here. 
\end{itemize}

\section{Some common problems}\label{sec-problems}
\setcounter{notecounter}{0}


\begin{itemize}
%
 \note{rememberstar} Remember \verb~*~'s for multiplication.  
  \begin{itemize}
   \item Remember especially that Maple converts
    \verb~2(x+y)~ to $2$: you probably meant $2*(x+y)$
    instead. 
   \item Enter $xy$ as \verb~x*y~; Maple will print this as
    $x\;y$, with a little space between the letters.  If you
    see $xy$ with no space, that means that Maple is
    referring to a single variable with the two-letter name
    ``$xy$''.  You probably typed \verb~xy~ instead of
    \verb~x*y~ by mistake. 
   \item For the product of $b$ and $x+y$, enter
    \verb~b*(x+y)~; Maple will print this as $b\;(x+y)$,
    with the $b$ in italics and followed by a little space. 
    If you see $\text{b}(x+y)$ (with an upright $b$ and no
    space) then Maple is referring to the value of a
    function called $b$ at $x+y$ (just as $\text{f}(x+y)$ is
    the value of the function $f$ at $x+y$).  You probably
    typed \verb~b(x+y)~ instead of \verb~b*(x+y)~, by
    mistake. 
  \end{itemize}
%
 \note{nostar} Remember not to use \verb~*~'s for application of
  functions. Things like $\sin x$ in traditional notation
  must be entered as \verb~sin(x)~, not \verb~sin*x~ or
  \verb~sin x~. 
%
 \note{fracbrac} $\frac{a+b}{c+d}$ must be entered as
  \verb~(a+b)/(c+d)~ (not \verb~a+b/c+d~), and
   $\frac{a}{(u+v)(x+y)}$ must be entered as
   \verb~a/((u+v)*(x+y))~ (not \verb~a/(u+v)*(x+y)~). 
%
 \note{expbrac} $a^{1/4}$ must be entered as \verb~a^(1/4)~, not
         \verb~a^1/4~. 
%
 \note{shape} Always use round brackets (not square or curly
         ones) for algebraic grouping.  Square brackets are
         used for lists or for subscripts (e.g. $x_5$ is
         \verb~x[5]~).  Curly brackets are used for sets. 
%
 \note{removedefs} Remember to remove definitions when you
  have finished with 
  them, by entering the \verb~restart;~ command.  If you
  enter an expression like $(x+y)^5$ and Maple converts it
  to something completely unrelated like
  $(\sin(\theta)+e^{-u})^5$, the most likely reason is that
  you previously defined \verb~x:=sin(theta);~ and
  \verb~y:=exp(-u);~, and you forgot to remove these
  definitions.  
%
 \note{badplotvar} 
  If you give $x$ a value and then try to use it as a
  plotting variable (e.g. by entering
  \verb~plot(x^2,x=-1..1);~) you will get the cryptic message
  ``Error, (in plot) invalid arguments''.  You should
  restart (or just enter \verb~unassign('x');~) and try again. 
%
 \note{badexp} Remember the constant $e=2.71828\dots$ must be
  entered as \verb~exp(1)~, and $e^x$ must be entered as
  \verb~exp(x)~.  Maple will print this as $\textbf{e}^x$,
  with a bold $e$.  If Maple is doing funny things with
  exponentials, look at the $e$'s in your expression.  If
  they are not bold, then you must have entered $e^x$ as
  \verb~e^x~ instead of \verb~exp(x)~, by mistake.  You can
  cure this by entering \verb~e:=exp(1);~. 
%
 \note{badpiagain} Remember that the constant
  $\pi=3.14159\dots$ must be 
  entered as \verb~Pi~, not \verb~pi~.  If Maple is doing
  funny things like refusing to simplify $\sin(\pi)$ to $0$,
  it may be that you entered \verb~pi~ instead of \verb~Pi~
  by mistake.  You can cure this by entering \verb~pi:=Pi;~. 
\end{itemize}

\section{Why the Classic Worksheet?}\label{sec-classic}

There are two different user interfaces to Maple: the ``Classic
Worksheet'' interface (with the orange icon) and the ``Standard''
interface (with the red icon).  We will use the Classic
interface, for several reasons:

\begin{itemize}
 \item The syntax used by the Classic interface is similar to that used
  in other contexts: programming languages such as Java or PHP,
  spreadsheet systems such as Excel, mathematical software such as
  Matlab, and so on.  It is therefore a good general skill to be
  familiar with this syntax.  By contrast, the skills that you need for
  the Standard interface are of little use for other systems.
 \item With the Classic interface you type a sequence of characters
  and they appear on the screen as you typed them.  This means that it
  is easy for the lecturer to explain what you need to type, and how
  to fix the most obvious problems.  The Standard interface involves
  much more clicking with the mouse and typing cursor and tab keys
  that are not directly visible on the screen.  This makes it much
  harder to provide comprehensible written instructions.
 \item The online tests also use the same syntax as the Classic
  interface (and it is not possible to change that).
 \item The Classic interface is much faster on less powerful computers.
\end{itemize}

You are free to experiment with the Standard interface if you wish to,
but you should remember that some points in the exam will rely on
knowledge of the Classic interface.

\end{document}
