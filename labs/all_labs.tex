\documentclass[a4paper,10pt]{amsbook}
%\usepackage{a4wide}
\usepackage{verbatim}
\usepackage{hyperref}

\setlength{\topmargin}{-1cm}
\setlength{\oddsidemargin}{-1cm}
\setlength{\evensidemargin}{-1cm}
\setlength{\textwidth}{18cm}
\setlength{\textheight}{24.5cm}

\newcommand{\N}         {{\mathbb{N}}}
\newcommand{\Z}         {{\mathbb{Z}}}
\newcommand{\Q}         {{\mathbb{Q}}}
\newcommand{\R}         {{\mathbb{R}}}
\newcommand{\C}         {{\mathbb{C}}}
\newcommand{\bbm}       {\begin{bmatrix}}
\newcommand{\bsm}       {\left[\begin{smallmatrix}}
\newcommand{\ebm}       {\end{bmatrix}}
\newcommand{\esm}       {\end{smallmatrix}\right]}
\newcommand{\bcf}[2]{\left(\begin{array}{c}{#1}\\{#2}\end{array}\right)}
\newcommand{\tm}        {\times}
\newcommand{\iffa}      {\Leftrightarrow}
\newcommand{\half}      {\tfrac{1}{2}}

\newcommand{\al}        {\alpha}
\newcommand{\bt}        {\beta}
\newcommand{\gm}        {\gamma}
\newcommand{\dl}        {\delta}
\newcommand{\ep}        {\epsilon}
\newcommand{\zt}        {\zeta}
\newcommand{\et}        {\eta}
\newcommand{\tht}       {\theta}
\newcommand{\io}        {\iota}
\newcommand{\kp}        {\kappa}
\newcommand{\lm}        {\lambda}
\newcommand{\ph}        {\phi}
\newcommand{\ch}        {\chi}
\newcommand{\ps}        {\psi}
\newcommand{\rh}        {\rho}
\newcommand{\sg}        {\sigma}
\newcommand{\om}        {\omega}

\newcommand{\vp}        {\mathbf{p}}
\newcommand{\vq}        {\mathbf{q}}
\newcommand{\vr}        {\mathbf{r}}
\newcommand{\Dl}        {\Delta}
\newcommand{\sm}        {\setminus}
\newcommand{\sse}       {\subseteq}
\newcommand{\st}        {\;|\;}
\newcommand{\xra}       {\xrightarrow}

\newcommand{\rr}        {\sqrt{3}}
\newcommand{\rs}        {\sqrt{6}}
\newcommand{\rt}        {\sqrt{2}}

\newcommand{\csch}     {\operatorname{csch}}
\newcommand{\sech}     {\operatorname{sech}}
\newcommand{\arcsinh}  {\operatorname{arcsinh}}
\newcommand{\arccosh}  {\operatorname{arccosh}}
\newcommand{\arctanh}  {\operatorname{arctanh}}

\newcommand{\degree}    {\operatorname{degree}}
\newcommand{\range}     {\operatorname{range}}
\newcommand{\trans}     {\operatorname{trans}}
\newcommand{\trc}       {\operatorname{trace}}
\newcommand{\adj}       {\operatorname{adj}}

\newcommand{\VEC}[1]    {\mathbf{#1}}

\newcommand{\RED}[1]{#1}
\newcommand{\OLIVEGREEN}[1]{#1}
\newcommand{\BLUE}[1]{#1}
\newcommand{\PURPLE}[1]{#1}
\newcommand{\DEFN}{\emph}
\newcommand{\PMA}{PMA}

\newcommand{\sect}{\section}

\newcommand{\papers}[1]{\textbf{[#1]}}

\renewcommand{\:}{\colon}




\numberwithin{example}{chapter}

\begin{document}

\chapter{Introduction}

\section{Getting started}

In working through this sheet, and the lab sheets in subsequent weeks,
you will need to consult the ``Maple reference'' notes.  If you see
something in the lab sheet like \note{8.6}, that means that item 8.6
in the Maple notes is relevant.

\section{Maple as a calculator}

\begin{exercise}\label{ex-twoplustwo}
 Enter
\begin{verbatim}
  2+2;
\end{verbatim}
 (As with almost all Maple commands, we need a semicolon at
 the end \note{\NOTEsemi}.  If Maple says ``Warning,
 premature end of input'', you probably left out the
 semicolon.)  Then press the ENTER key.  Maple will respond
 with
 \[ 4 \]
 Now try the following calculation%
 \footnote{From the Beaver's Lesson:
 \verb~http://tinyurl.com/2dv793~,
 verses 16--17}:
\begin{verbatim}
  (3 + 7 + 10) * (1000 - 8) /(900 + 90 + 2) - 17;
\end{verbatim}
 Note that we use \verb~*~ for
 multiplication\note{\NOTEstar}; it is a very common error to 
 leave this out.  If you ever see a message saying
 \begin{quotation}
  Error, missing operator or `;` 
 \end{quotation}
 then the most likely explanation is a missing \verb~*~.  
\end{exercise}

\begin{exercise}\label{ex-numeric-misc}
 Try the following calculations.  In each case the answer will have
 some kind of pattern to it; see if you can explain it. 
 \begin{itemize}
  \item[(a)] $6^{20}\;15^{20}/9^{20}$ (Enter this as
   \verb~6^20*15^20/9^20;~) 
  \item[(b)] $(10^{10}-1)/99$
  \item[(c)] $(10^{10}-10-9^2)/9^2$ (The explanation here is
   harder, and should be considered a challenge for enthusiasts only.)  
  \item[(d)] $(10^9+1)(10^{10}-10-9^2)/9^2$ (You should be able to
   explain how this relates to~(c), even if you cannot explain~(c)
   itself.)  (The correct answer is \emph{not}
   $1000000001/81$; if you get that answer, you probably
   left out a \verb~*~.) 
 \end{itemize}
\end{exercise}

Maple will usually leave answers as fractions (or other exact
mathematical expressions) rather than giving an approximate numerical
answer.  If you want a numerical answer, you can use the function
\verb~evalf()~.  For example:
\begin{verbatim}
 1 + 1/2 + 1/3 + 1/4;
\end{verbatim}
\[ 25/12 \]
\begin{verbatim}
 evalf(1 + 1/2 + 1/3 + 1/4);
\end{verbatim}
\[ 2.083333333 \]
(A slightly more efficient approach here is to enter
\verb~1+1/2+1/3+1/4;~, and then \verb~evalf(%);~.  The
\verb~%~ symbol refers to the last thing that Maple worked
out \note{\NOTEpercent}.) 

\begin{exercise}\label{ex-cfrac-pi}
 Calculate the following as exact fractions, then as numerical
 approximations.  What do you see? 
 \[ \text{(a)}\hspace{1em}
     3+\cfrac{1}{7}
    \hspace{3em}
    \text{(b)}\hspace{1em}
     3+\cfrac{1}{7 + \cfrac{1}{15}}
    \hspace{3em}
    \text{(c)}\hspace{1em}
     3+\cfrac{1}{7 + \cfrac{1}{15 + \cfrac{1}{1 + \cfrac{1}{293}}}}  
  \]
  You need to be very careful with brackets when entering these
  expressions.  For example, the last one should be
\begin{verbatim}
  3 + 1/(7 + 1/(15 + 1/(1 + 1/293)));
\end{verbatim}
\end{exercise}

\begin{exercise}\label{ex-e-digits}
 Maple's notation for the constant $e\simeq 2.718281828$ (the base of
 natural logarithms) is \verb~exp(1)~  (see
 \note{\NOTEexp}).  Read \note{\NOTEevalfdigits,\NOTEdigits} to find out
 how to calculate $4997$ digits of $e$.  (You should notice
 that the last $5$ digits are all the same; this is the first time
 that this happens.) 
 \footnote{You may think that calculating thousands of digits of $e$
 is a useless activity, but it could get you a job at Google: see
 \verb~http://tinyurl.com/ygat93~
 }
\end{exercise}

\begin{exercise}\label{ex-near-int}
 It is a remarkable fact that $e^{\pi\sqrt{163}}$ is extremely close
 to being an integer (for a reason involving some very advanced
 mathematics).  Let us check this. 
 \begin{itemize}
  \item[(a)] Tell Maple to do all calculations to $40$
   digits (see \note{\NOTEdigits}).
  \item[(b)] Enter \verb~x:=exp(Pi*sqrt(163));~.  Maple will
   display this in traditional notation, but will not evaluate it
   numerically.  For the syntax used here, see
   \note{\NOTEcolonequals,\NOTEexp,\NOTEpi,\NOTEsqrt}. 
  \item[(c)] Enter \verb~y:=evalf(x);~.  Maple will print this in
   scientific notation, which is not too helpful for our present
   purposes.  To display it in a more useful way, enter
\begin{verbatim}
  printf("%20.20f\n\n",y);
\end{verbatim}
   (Do not worry about the details of this, as we will not
   use the \verb~printf~ command again.)  You should see
   that there is a long string of $9$'s after the decimal
   point, showing that $y$ is just a tiny bit less than an
   integer. 
  \item[(d)] To find the integer nearest to $y$, enter
   \verb~z:=round(y);~.  (Of course this can just be read off
   from~(c), but we prefer not to have to retype it.)  Now calculate
   $y-z$, to see that $y$ is less than $10^{-12}$ away from an
   integer. 
  \item[(e)] Enter \verb~restart;~.  This removes the definitions of
   $x$, $y$ and $z$, which would otherwise cause trouble in later
   exercises. 
 \end{itemize}
\end{exercise}

\section{Symbolic algebra}

\begin{exercise}\label{ex-horner}
 Enter
\begin{verbatim}
  A := (x^2-4*y^2)*(x^3-x*y^2);
\end{verbatim}
 (If Maple prints this as a horrible expression involving
 $e^{\pi\sqrt{163}}$, then you probably forgot to remove the
 definitions of $x$, $y$ and $z$ made in the previous exercise.  Enter
 \verb~restart;~ and try again.) 
 
 Now try \verb~simplify(A);~, \verb~expand(A);~, \verb~factor(A);~ and
 \verb~convert(expand(A),horner,x);~.  Which of these forms do you think is
 the most illuminating?  (There is no right answer.) 
\end{exercise}

\begin{exercise}\label{ex-rat-misc}
 Ask Maple to simplify the expression 
 \[ \frac{2x}{x^2-1} + \frac{1}{x+x^2} + \frac{1}{x-x^2}. \]
 Now see if you can get the same answer by hand. 
\end{exercise}

\section{Plotting}

\begin{exercise}\label{ex-sin-taylor}
 Ask Maple to plot the graph of $y=\sin(x)$ from $x=-4$ to
 $x=4$ (see \note{\NOTEbasicplot} for the syntax).  Now plot
 the function
 \[ y = x-x^3/6+x^5/120-x^7/5040 \]
 from $x=-4$ to $x=4$.  What do you notice?  Plot the two curves
 together, as explained in \note{\NOTEtwoplots}.
\end{exercise}

\section{Solving}

\begin{exercise}
 Ask Maple to solve the equations 
 \begin{align*}
  x^2 + y^2 &= 1 \\
  (x-1)^2 + (y-1)^2 &= 1,
 \end{align*}
 like this:
\begin{verbatim}
  solve({x^2+y^2=1,(x-1)^2+(y-1)^2=1},{x,y});
\end{verbatim}
 Can you find the solution by hand?  Can you draw (by hand)
 the curves $x^2+y^2=1$ and $(x-1)^2+(y-1)^2=1$, and thus
 obtain the solution graphically?
\end{exercise}

\section{Calculus}

\begin{exercise}
 Read \note{\NOTEdiff,\NOTEdiffvar}.  Then ask Maple to
 differentiate the function $\ln(\ln(\ln(x)))$ with respect
 to $x$.  Now differentiate the functions $(3x+4)/(2x+3)$
 and $(1+x^2+x^4/2)e^{-x^2}$, and simplify your answers.
\end{exercise}

\begin{exercise}
 Ask Maple to evaluate the integral
 $\int_0^1\sqrt{1-x^2}\,dx$, as explained in \note{\NOTEdefint}
\end{exercise}

\chapter{Solving equations} 

Maple has powerful facilities for solving equations, but they are not
perfect.  Often one has to rearrange an equation in some way before
Maple will succeed in solving it.  It is quite common for Maple to
give a complicated solution, when a slightly different approach might
give a simpler one.  Thus, solving equations is more of an art than a
science.  

Numbers in square brackets refer to the ``Maple
reference'' notes, which were distributed in the first lecture.

\section{Algebraic equations}

\begin{exercise}\label{ex-decimal-roots}
 Enter
\begin{verbatim}
 f := (x) -> 2*x^4-2222*x^3+224220*x^2-2222000*x+2000000;
\end{verbatim}
 to define the function
 \[ f(x) = 2x^4-2222x^3+224220x^2-2222000x+2000000. \]
 (See \note{\NOTEarrow -- \NOTEshowfuncdef} for more about
 the syntax used here for defining functions.) Then solve 
 $f(x)=0$ (as explained in \note{\NOTEsolve})
 to find the roots.  Use the answer to factorize $f(x)$, and then
 enter \verb~factor(f(x));~ to check your answer.
\end{exercise}

\begin{exercise}\label{ex-biquadratic-roots}
  Define $y=x^4-x^3-x^2-x/8+1/64$ and $z=16x^3-24x^2-6x+2$.  (Here we
  do not use the arrow notation, because we do not have a $(x)$ on the
  left hand side.)  Ask Maple to find the values of $x$ where $y=0$;
  you should get four solutions.  It is best to use a command like
  \verb~sols := solve(...)~ so that you can reuse the solutions later;
  now \verb~sols[1]~ will refer to the first solution, \verb~sols[2]~
  will refer to the second, and so on.  Now look more closely at the
  solutions.  You should see that they all involve the same four
  terms, but with different plus and minus signs.  Examine the pattern
  of signs carefully.  Can you rephrase Maple's answer more neatly?
  Next, read \note{\NOTEsubs -- \NOTEsolvesubs}.  Use the
  \verb~subs()~ command to find the value of $z$ at the four points
  where $y$ is zero.  You can do this as four separate steps, or
  (better) you can read \note{\NOTEseq} and use the \verb~seq()~
  command to do it in one go.  (There are some interesting phenomena
  going on in this calculation, that will eventually be explained in
  the level 4 course on Galois Theory.)
\end{exercise}

\begin{exercise}\label{ex-cubic-horrid}
 As general background, you should be aware that the last two
 exercises were chosen carefully to work out nicely.  For a typical
 polynomial there may be no formula for the roots, and even if there
 is, it may be very ugly.  Moreover, even if there is a formula and
 the roots are all real numbers, the formula may still involve complex
 numbers as an intermediate stage.  For an example of this, set
 \verb~_EnvExplicit:=true;~ again, and then ask for the roots of
 $x^3-3x+1$.  Notice that the answer involves $I$ (the square root of
 minus one) in several places.

 In a case like this, it is more useful to find numerical
 approximations to the roots instead of a massive exact
 formula.  Read \note{\NOTEfsolve}, and ask Maple to find an
 approximate solution to the equation $y=0$, where $y=x^3-3x+1$.
 It is also helpful to plot the graph (say for $-2\leq x\leq 2$) and
 see where the roots lie: read \note{\NOTEbasicplot} for help with
 this.  From either of these approaches, we see that there is
 precisely one negative root for this equation.  Use \note{\NOTEdiff}
 and \note{\NOTEsubs} to find the value of $dy/dx$ at this negative
 root. 
\end{exercise}

\begin{exercise}\label{ex-mobius-fix}
 Enter the definition
 \[ g(x) = \frac{b^2-c^2+(1+c^2)x}{1-c^2+ c^2x}. \]
 We will study the fixed points of $g$, or in other words the values
 where $g(x)=x$.  
 \begin{itemize}
  \item[(a)] Ask Maple to solve the equation $g(x)=x$ for
   $x$.  (Remember to use the syntax in \note{\NOTEarrow}
   when entering the definition of $g$.)  You should get two
   solutions. 
  \item[(b)] For some special value(s) of $b$ and/or $c$, Maple's
   answer does not make sense.  Go back to the definition of $g$ and
   the equation $g(x)=x$, and work out by hand what happens in those
   special cases.
  \item[(c)] For which value(s) of $b$ and/or $c$ are Maple's two
   solutions actually the same?
  \item[(d)] Give a self-contained summary of your conclusions, that
   could be read and understood by someone who had not attempted the
   question. 
 \end{itemize}
\end{exercise}

\section{Approximate solutions}

\begin{exercise}\label{ex-adjusted-sin}
 Define $f(x)=\sin(\pi x+e^{-x})$ (remembering that $e^{-x}$ is
 \verb~exp(-x)~ and $\pi$ is \verb~Pi~).  If you ask Maple to solve
 this, it will give an answer in terms of an obscure function called
 \verb~LambertW~.  \footnote{If you are curious, you can enter
   \verb~?LambertW~ or visit
   \verb~http://mathworld.wolfram.com/LambertW-Function.html~.} 
 We will ignore this for the moment, and look instead for a numerical
 approximation to the roots, concentrating on the case where
 $x\geq 0$.
 \begin{itemize}
  \item[(a)] First plot the graph of $f(x)$, say from $x=0$ to $x=10$. 
   Roughly where are the roots?  Can you explain why they are where
   they are? 
  \item[(b)] Enter \verb~fsolve(f(x)=0,x);~.  This finds a root at
   about $x=-0.55$, ignoring all the roots with $x\geq 0$ that we saw
   in the graph.  To find a root near $x=2$ instead, enter
\begin{verbatim}
  fsolve(f(x)=0,x=2);
\end{verbatim}
   To find roots near $x=1$, $x=2$, $x=3$ and so on, up to $x=10$,
   enter this:
\begin{verbatim}
  seq(fsolve(f(x)=0,x=n),n=1..10);
\end{verbatim}
  \item[(c)] Now define $r(n)=n-e^{-n}/\pi-e^{-2n}/\pi^2$
   (using syntax as in \note{\NOTEseriesdef}).  In~(a)
   and~(b) we saw that for every integer $n$, there is a root that is
   close to $x=n$.  I claim that this root is even closer to
   $x=r(n)$.  To see this, enter the definition of $r$ and then 
\begin{verbatim}
  seq(evalf(r(n)),n=1..10);
\end{verbatim}
   Compare this with your final answer in~(b). 
 \end{itemize}
\end{exercise}

\section{Infinite families of solutions}

\begin{exercise}
 Consider the equation $\tan(\pi x)^2=3$.  
 \begin{itemize}
  \item[(a)] To find the solutions graphically, plot the
   function $\tan(\pi x)^2-3$ for a reasonable range of
   values of $x$.  (Because $\tan(\pi x)$ blows up to
   infinity for certain values of $x$, it is necessary to cut
   down the vertical range to get a meaningful picture
   \note{\NOTEvertrange}, and it is also useful to ask Maple
   to calculate extra points to make the graph more accurate
   \note{\NOTEnumpoints}.  You should see that there are many
   roots in the picture, repeating in a regular way, so there
   are actually infinitely many roots if we allow $x$ to run
   from $-\infty$ to $+\infty$.
  \item[(b)] Click with your mouse on each of the places where the
   graph crosses the $x$-axis; approximate coordinates will then be
   shown in a small box at the top left of the Maple window.  You
   should be able to guess the exact coordinates from this. 
  \item[(c)] Ask Maple to solve the equation $\tan(\pi x)^2=3$ for $x$. 
   Maple will report only two solutions, although we have seen that
   there are really infinitely many. 
  \item[(d)] You should have seen in~(b) that the solutions are all of
   the form $x=n+1/3$ or $x=m-1/3$, where $n$ and $m$ are integers. 
   To persuade Maple to find this answer, we have to enter
   \verb~_EnvAllSolutions:=true;~, and then ask Maple again to solve
   the equation.  It will respond as follows:
   \[ \left\{ x=1/3+{\it \_Z1} \right\} ,
      \left\{ x=-1/3+{\it \_Z2} \right\} 
   \]
   This is the same as our answer, except that Maple uses the symbols 
   $\it \_Z1$ and $\it \_Z2$ (instead of $n$ and $m$) for arbitrary
   integers.  
 \end{itemize}
\end{exercise}

\section{Linear equations}
 
\begin{exercise}\label{ex-hilbert-lineq}
 Consider the equations
  $x/2 + y/3 + z/4 = 1$, \hspace{2em}
  $x/3 + y/4 + z/5 = 2$, \hspace{2em}
  $x/4 + y/5 + z/6 = 3$.\\
 To solve these, we enter
\begin{verbatim}
  eqns := {x/2 + y/3 + z/4 = 1,x/3 + y/4 + z/5 = 2,x/4 + y/5 + z/6 = 3};
  sols := solve(eqns,{x,y,z});
\end{verbatim}
 It would also work to do all this in one step:
\begin{verbatim}
  solve({x/2+y/3+z/4=1,x/3+y/4+z/5=2,x/4+y/5+z/6=3},{x,y,z});
\end{verbatim}
 However, this can make things cramped and hard to organise,
 especially if our equations are just one piece of a more complex
 problem.  Now find the value of $x^2+y^2+z^2$ at the point where the
 above equations are satisfied.
\end{exercise}

\begin{exercise}\label{ex-linear-trichotomy}
 Solve the following systems of equations by hand.  You should find
 that 
 \begin{itemize}
  \item one system has no solutions
  \item one system has a single, fully-determined solution
  \item one system has a solution in which one of the variables is
   free to take any value; this means that there are infinitely many
   different solutions. 
 \end{itemize}
 \begin{itemize}
  \item[(a)] $p +  q +  r = 0$ \hspace{3em}
             $p + 2q + 3r = 1$
  \item[(b)]
    $u + v  = 1001$,  \hspace{2em} 
    $u + 2v = 1002$,  \hspace{2em} 
    $u + 3v = 1006$
  \item[(c)] 
      $x +  y +  z = 2$,  \hspace{2em} 
      $x + 2y + 3z = 2$,  \hspace{2em} 
      $x + 4y + 9z = 2$
 \end{itemize}
 When you have found the solutions by hand, find them again
 using Maple \note{\NOTEsolve,\NOTEmultisolve}.  Note that
 Maple returns an equation like $w=w$ to indicate that $w$
 may take any value.  Note also that when there are no
 solutions, Maple prints nothing at all, which can be
 disconcerting.  If this worries you, you can enclose the
 \verb~solve~ command in square brackets.  For example,
 \verb~solve({x=1,2*x=1},{x});~ gives a completely empty
 response, but \verb~[solve({x=1,2*x=1},{x})];~ gives a
 response of $[]$, so you can at least see that Maple has
 actually done something.
\end{exercise}

\chapter{Plotting 1}

Before starting each exercise, you should restart Maple, either by
entering the \verb~restart;~ command, or by clicking the button with
the circulating arrow at the right hand end of the toolbar.

Numbers in square brackets refer to the ``Maple
reference'' notes, which were distributed in the first lecture.

There are some hints at the end.

\begin{exercise}\label{ex-plot-misc}
 Plot each of the following functions \note{\NOTEbasicplot}
 for a suitable range of values of $x$.  Experiment to find
 a range that displays the interesting features.  Write two
 or three lines (for each function) describing those
 features.
 \begin{itemize}
  \item[(a)] $2e^{-t}\sin(30 t)$ 
  \item[(b)] $2\sin(20 t)+3\sin(21 t)$
  \item[(c)] $\sin(x)+\sin(3x)/3+\sin(5x)/5+\sin(7x)/7$
  \item[(d)] $\cos(\pi x^2)$
 \end{itemize}
\end{exercise}

\begin{exercise}\label{ex-wiggle}
 Consider the function 
 \[ f(x) = (x^3-x)(x^2-4/9)(x^2-1/9). \] Enter this
 definition, using the syntax explained in
 \note{\NOTEarrow}.  Plot the graph for various ranges of
 $x$, and describe the main features that you see.  Compare
 $f(x)$ with the functions $x$, $x^2$, $x^3$ and so on, by
 plotting them together \note{\NOTEtwoplots}.  Which of
 these matches $f(x)$ most closely for large $x$?  Can you
 explain why?  
\end{exercise}

\begin{exercise}\label{ex-mobius}
 Consider the function
 \[ g(x) = (2x+3)/(3x-4). \]
 \begin{itemize}
  \item[(a)]
   Plot the graph for various ranges of $x$, and describe the main
   features that you see.  You may find it helps to restrict the
   vertical range \note{\NOTEvertrange} as well as the horizontal one.
  \item[(b)] What is the value of $x$ where $g(x)$ is discontinuous? 
   You can either work this out from the formula, or read this off
   from the graph, or use the Maple command \verb~discont(g(x),x)~. 
   You can also replot the graph, skipping over the
   discontinuity, as explained in \note{\NOTEdiscont}.
  \item[(c)] 
   Plot the line $y=5$ along with the function $g(x)$  
   \footnote{You should use the syntax in \note{\NOTEtwoplots}.  You
   should also remember that the command to plot $y=5$ is
   \verb~plot(5,...)~, not \verb~plot(y=5,...)~.}. 
   You should see that the line crosses the graph.  If you change the
   $5$ to $-1$ and redo the plot, then again you see that the line
   $y=-1$ crosses the graph.  However, there is one horizontal line
   (somewhere between $y=-1$ and $y=5$) that does not cross the
   graph.  Can you find out which line it is?  You can either use
   plotting and trial and error, or analyse the situation
   algebraically. 
  \item[(d)] Enter \verb~limit(g(x),x=infinity);~ and
   \verb~limit(g(x),x=-infinity);~.  How are these related to
   part~(c)? 
 \end{itemize}
\end{exercise}
        
\begin{exercise}\label{ex-lhopital}
 Consider the function
 \[ f(x) = \frac{1 - e^x + e^{2x} - e^{3x}}{\sin(x)}. \]
 (Remember \note{\NOTEarrow} when entering this in Maple.)
 What is $f(0)$?  The formula above gives $f(0)=(1-1+1-1)/0=0/0$,
 which is meaningless.  However, if we plot the graph of $f(x)$ (for
 $x$ from $-1$ to $1$, say) we see that there is a perfectly definite
 value for $f(0)$; what is it?  
(See the hints at the end for a more careful discussion of the logic.) 
\end{exercise}

\begin{exercise}\label{ex-bessel}
 We now use Maple to investigate the properties of a new function that
 you probably have not met before: the Bessel function $J_2(x)$.  This
 is one of a whole family of Bessel functions, which have many
 applications, for example in studying the vibration of drums or the
 behaviour of fibre-optic cables.  In Maple it is called
 \verb~BesselJ(2,x)~. 
 \begin{itemize}
  \item[(a)] Plot $J_2(x)$ for $-50\leq x\leq 50$.  Describe the
   main features. 
  \item[(b)] Plot $J_2(x)$ and $x^2/5$ in the same picture, from $x=-0.1$
   to $x=+0.1$.  Now change the $5$ to something else, and repeat. 
   Which value makes the two curves match up most closely? 
  \item[(c)] Plot $J_2(x)$ from $x=10$ to $300$.  You should see some
   strange wiggles in the graph, which are in fact not really there;
   they appear because Maple has not calculated enough points to draw
   an accurate picture.  To fix this, we use the \verb~numpoints~
   option (as in \note{\NOTEnumpoints}):
\begin{verbatim}
  plot(BesselJ(2,x),x=10..300,numpoints=200);
  pic := %:
\end{verbatim}
   (The second line here saves the picture, so we do not have to
   recalculate it later \note{\NOTEsaveplot}.) 

   You should see an oscillation dying slowly away.  This leads us to
   ask how large the oscillations are, and what is their frequency. 
  \item[(d)] For the size of the oscillations, try plotting $0.15x^{-1}$
   alongside $J_2(x)$, as in \note{\NOTEdisplay}:
\begin{verbatim}
  plots[display](pic,plot(0.15*x^(-1),x=10..300,color=blue));
\end{verbatim}
   Do the same with $0.3x^{-1/3}$ and $0.5x^{-1/2}$ and some other
   similar functions.  (Do not retype the whole line; just edit the
   relevant numbers, press ENTER, and Maple will redraw the graph.) 
   If you get the numbers right, then the blue curve will just touch
   the tops of all the waves.  (In fact it does not touch exactly, but
   you need to zoom in very close to see that.) 
  \item[(e)] For the frequency of the oscillations, try plotting
   $J_2(x)$ alongside $\sin(x-\pi/4)$, for various ranges of $x$. 
   This should convince you that $J_2$ oscillates with approximately
   the same frequency as $\sin(x)$, at least when $x$ is reasonably
   large. 
  \item[(f)] Can you combine~(d) and~(e) to find an easier function
   function $f(x)$ that is very close to $J_2(x)$ for large $x$?  Plot
   $f$ and $J_2$ together to check your answer. 
 \end{itemize}
\end{exercise}

\begin{exercise}\label{ex-quartic}
 Put 
 \[ g(a,x) =
    \frac { (x-1-\sin(a/4))(x-1-\cos(a/4))
            (x+1-\sin(a/4))(x+1-\cos(a/4))}
          {1+{x}^{4}}. 
 \]
 (Remember \note{\NOTEarrow,\NOTEmultifunc} when entering
 this in Maple.)  The object of this exercise is to describe
 the properties of $g$.  
 \begin{itemize}
  \item[(a)] Plot $g(2,x)$ for $x$ from $-500$ to $500$.
   Write two or three lines describing the main features of
   the plot.  Then plot $g(3,x)$ and $g(4,x)$; you should see
   that on this scale, they look almost exactly the same.
  \item[(b)] Plot $g(2,x)$ and $g(4,x)$ for $x$ from $-2$ to $2$.  In
   both cases, the curve dips below the $x$ axis.  Now plot $g(3,x)$
   instead.  It appears that the curve does not dip below the axis,
   but just touches it somewhere near $x=-0.3$, and again near
   $x=1.7$.  To investigate further, we zoom in.  We plot only from
   $x=-0.4$ to $x=-0.2$, to focus on the region of interest.  We also
   restrict the vertical range \note{\NOTEvertrange} to be
   from $-0.01$ to $0$, so only the part of the curve below
   the axis will be shown.  Finally, we specify
   \verb~numpoints=1000~ to make sure that Maple plots the
   graph very accurately \note{\NOTEnumpoints}.  

%   The command is as follows:  
%\begin{verbatim}
%   plot(g(3,x),x=-0.4..-0.2,-0.01..0,numpoints=1000);    
%\end{verbatim}
   We see that the graph still dips slightly below the axis.
   However, there is a certain number $a$ close to $3$ for
   which the curve does not dip below the axis at all ---
   see if you can find it.  Plot the graph for various
   different values of $a$, zooming in further if necessary,
   and also thinking about which values of $a$ might make
   something special happen in the formula for $g(a,x)$.
  \item[(c)] Plot $1-8g(a,0)$ for a reasonable range of
   values of $a$.  You should recognize the resulting graph,
   and so should be able to give a simple formula for
   $1-8g(a,0)$.  Can you derive this formula algebraically
   from the definition of $g(a,x)$?
  \item[(d)] Plot $g(a,0.5)$ for $a$ from $-40$ to $40$.
   What are the main features?  You should see three tall
   peaks and three lower peaks.  Consider the tall peak
   closest to the $x$-axis.  Click on it and copy down the
   coordinates that you see in the little box at the top
   left of the Maple window.  They should be $(3,0.9)$
   approximately.  To get more accurate numbers, recall that
   the peak occurs where the derivative of the function is
   zero.  Here we are thinking of $g$ as a function of $a$,
   so the slope is $dg/da$, or in Maple terms:
\begin{verbatim}
    slope := diff(g(a,0.5),a);
\end{verbatim}
   We need to solve numerically for the place where the
   slope is zero, close to $a=3$:
\begin{verbatim}
    fsolve(slope = 0,a=3);
\end{verbatim}
   You should recognize the answer.  Now find the
   $x$-coordinates of the other two tall peaks in the same
   way.  Can you give exact formulae for these as well?
   (Hint: divide them by the $x$-coordinate of the first
   peak.)
  \item[(e)] Now try doing some $3$-dimensional plots, for
   example
\begin{verbatim}
   plot3d(g(a,x),a=-20..20,x=-15..15,axes=boxed,grid=[100,100]);
\end{verbatim}
 \end{itemize}
\end{exercise}

\section*{Hints}

\subsection*{Exercise~\ref{ex-lhopital}}
\begin{verbatim}
 f := (x) -> (1 - exp(x) + exp(2*x) - exp(3*x))/sin(x);
 plot(f(x),x=-1..1);
\end{verbatim}
The graph crosses the $y$-axis at $y=-2$, suggesting that $f(0)=-2$. 

If we want to be perfectly logical, we should say something a little
bit different, however.  This is our function $f$, so it is our right
and privilege to define it however we want.  The original formula does
not make sense for $x=0$, so $f(0)$ is not yet defined.  We can, if we
choose, declare that $f(0)$ is simply undefined.  Alternatively, we
can, if we feel perverse, declare that $f(0)=42$.  Alternatively, we
can declare that $f(0)=-2$.  The graph shows only that the last option
is the most reasonable one, not that we are forced to take it. 

\subsection*{Exercise~\ref{ex-bessel}}
\begin{itemize}
 \item[(b)] \verb~plot({BesselJ(2,x),x^2/5},x=-0.1..0.1);~\\
  The best match occurs with $x^2/8$ instead of $x^2/5$. 
 \item[(d)] The graph of the function $y=0.8x^{-1/2}$ touches the tops
  of the waves (not exactly, but to a very good approximation). 
 \item[(f)] The function $J_2(x)$ is very close to
  $-0.8\sin(x+\pi/4)/\sqrt{x}$ (for large $x$).  Enter the following
  to plot them together:
\begin{verbatim}
  plot({BesselJ(2,x),-0.8*sin(x+Pi/4)/sqrt(x)},x=1..40);
\end{verbatim}
\end{itemize}

\subsection*{Exercise~\ref{ex-quartic}}
The definition of $g$ is
\begin{verbatim}
  g := (a,x) -> (x-1-sin(a/4))*(x-1-cos(a/4))*
                (x+1-sin(a/4))*(x+1-cos(a/4))/(1+x^4);
\end{verbatim}

\chapter{Plotting 2}

Before starting each exercise, you should restart Maple, either by
entering the \verb~restart;~ command, or by clicking the button with
the circulating arrow at the right hand end of the toolbar.  After
restarting Maple, enter \verb~with(plots):~ to (re)load the
\verb~plots~ package.  (It is safest to do this for all questions,
although it is unnecessary for some.  If the \verb~display~ command
gives a big mess of coordinates instead of a picture, you probably
forgot to enter \verb~with(plots):~.  Any warning messages produced by
this command can safely be ignored.)

\section{Parametric plotting}

So far we have mostly drawn graphs where $y$ is given as a function of
$x$.  We now instead draw some graphs where both $x$ and $y$ are given
as functions of another variable, say $t$.

\begin{exercise}
 \begin{itemize}
  \item[(a)]
   Plot the curve given by $x=\cos(10t)/(1+t^2)$ and $y=\sin(10t)/(1+t^2)$,
   as in \note{\NOTEparametric,\NOTEparametricbrackets}.
  \item[(b)]
   Plot the curve $(x,y)=(\sin(3t),\sin(2t))$ for $t$ from $0$ to
   $2\pi$ (this is called a Lissajous figure).  Replace the $2$ and
   $3$ by larger numbers, and investigate how the picture changes.
  \item[(c)]
   Plot the curve $(x,y)=(t-\sin(t),1-\cos(t))$ for $t$ from $0$ to
   $8\pi$.  This is called a cycloid; it is the path traced out by a
   point on the edge of a wheel as the wheel rolls along the ground.
   It looks wrong because Maple draws it using different scales on the
   $x$ and $y$ axes.  To fix this, click on the graph, and then click
   on the button marked ``1:1'' on the toolbar.  You can click the
   button repeatedly to switch between the distorted and undistorted
   pictures.  Alternatively, use the option \verb~scaling=constrained~
   when you first draw the graph \note{\NOTEscaling}, like this:
\begin{verbatim}
  plot([t-sin(t),1-cos(t),t=0..8*Pi],scaling=constrained);
\end{verbatim}
  \item[(d)] Plot the curve
   $(x,y)=(2t/(1+t^2),(1-t^2)/(1+t^2))$, again using
   \note{\NOTEparametric}.  What do you see?  Can you
   explain it?
 \end{itemize}
\end{exercise}

\begin{exercise}\label{ex-nodal}
 Consider the curve $(x,y)=(4t^2-1,8t^3-8t)$, which is called a nodal
 cubic.  In this exercise it will be convenient to enter these
 definitions separately from the \verb~plot~ command:
\begin{verbatim}
  x := 4*t^2-1;
  y := 8*t^3-8*t;
\end{verbatim}
 (You must remember to remove these definitions by restarting Maple
 \note{\NOTErestart} before you go on to the next exercise.) 
 \begin{itemize}
  \item[(a)] Plot the curve from $t=-1.5$ to $t=1.5$:
   \verb~plot([x,y,t=-1.5..1.5]);~  
  \item[(b)] Notice that the curve crosses over itself.  What are the
   $x$ and $y$ coordinates of the point where this happens, and what
   are the corresponding values of $t$? 
  \item[(c)] The curve crosses the $y$-axis twice.  What are the
   values of $t$ and $y$ at these two points? 
  \item[(d)] Now plot the function $(X-3)\sqrt{X+1}$ for $X$ from $-1$
   to $8$.  (Maple will not let us use $x$ here because
   $x$ has been defined in terms of $t$ and so is no longer a free
   variable \note{\NOTEbadplotvar}.  We could use almost any
   other letter, but $X$ seems a  natural choice.)  To
   compare this precisely with~(a), combine the two pictures
   like this \note{\NOTEdisplay,\NOTEparametricrange}: 
\begin{verbatim}
  display(
    plot((X-3)*sqrt(X+1),X=-1..8,color=blue),
    plot([x,y,t=-2..2],color=red),
    view=[-2..8,-15..15]     
  );
\end{verbatim}
   (If this gives a big mess of coordinates instead of a picture, you
   probably forgot to enter \verb~with(plots):~.) 
   What do you observe?  Can you test it algebraically? 
 \end{itemize}
\end{exercise}

\begin{exercise}
 \textbf{Do not enter} the following commands.  Instead, read them
 carefully, study the reference notes, and sketch what Maple would
 plot if you entered the commands.
 \begin{itemize}
  \item[(a)] \verb~plot([sin(x),cos(x),x=0..4*Pi]);~
  \item[(b)] \verb~plot([sin(x),cos(x)],x=0..4*Pi);~
  \item[(c)] \verb~plot([x,sin(x),x=0..4*Pi]);~
  \item[(d)] \verb~plot([y,sin(y),y=0..4*Pi]);~
  \item[(e)] \verb~plot([sin(y),y,y=0..4*Pi]);~
  \item[(e)] \verb~plot([sin(y),y],y=0..4*Pi);~
 \end{itemize}
\end{exercise}

\section{Implicit plotting}


Often we want to plot a curve given by an equation like
$x^2+y^2=100$, where $y$ is not explicitly a function of
$x$.  Here we might let $x$ and $y$ run from $-11$ to $11$.
Read \note{\NOTEimplicitplot} and ask Maple to plot the
graph. 

\begin{exercise}
 Plot the curve $y^2=x^3-x$, for a range of values of $x$
 and $y$ that shows the interesting features.  The
 \verb~implicitplot~ command will give you a rather jagged
 picture; you can improve it as in \note{\NOTEgrid}.

 Now plot $y^2=x^3-x+a$ for various values of $a$ between
 $0$ and $1$.  You should see that at a certain value of
 $a$, the picture changes from being two separate curves to
 a single, connected curve.  Find the relevant value of $a$
 approximately, by trial and error.  Can you work out an
 exact formula? 
\end{exercise}

\begin{exercise}
 Plot the curve $x^2+y^2+a(\sin(2\pi x)+\cos(2\pi y))=100$ for
 various values of $a$, starting with $a=1$, $a=5$ and
 $a=20$.  You will need lots of points to get a good
 picture, so use the option \verb~grid=[200,200]~.  Describe
 the main features.  (I have not yet  worked out how to
 explain them; you can take that as a challenge!) 
\end{exercise}

\section{Plotting lists of points}

\begin{exercise}\label{ex-listplot}
 Read \note{\NOTElistplot\dots\NOTElistplotgen}.  Ask Maple
 to draw the seven points at $x=1,\dotsc,7$ with $y$-values
 $10$, $40$, $20$, $30$, $50$, $10$ and $20$.  Do this with
 a line joining the specified points \note{\NOTElistplot}
 and then with just the points themselves
 \note{\NOTElistplotstyle}.  Try changing the list of
 $y$-values in various ways and redrawing the graph.
\end{exercise}

\begin{exercise}\label{ex-plotprimes}
 \begin{itemize}
  \item[(a)] Enter \verb~ithprime(1);~, \verb~ithprime(2);~ and so on
   to list the first few prime numbers. 
  \item[(b)] Enter \verb~seq(ithprime(i),i=1..100);~ to list the first
   hundred primes in one go \note{\NOTEseq}. 
  \item[(c)] To plot this list, enter 
\begin{verbatim}
  listplot([seq(ithprime(i),i=1..100)],style=POINT);
\end{verbatim}
   Note that the \verb~seq~ command gives a list with no
   brackets, but the \verb~listplot~ command needs a list
   with square brackets, so we have put square brackets
   around the \verb~seq~ command.

   To save this picture for future use, enter 
\begin{verbatim}
  pic1 := %:
\end{verbatim}
   \textbf{Note} that this ends with a colon, not a semicolon.  If you
   use a semicolon, then Maple will print out all the coordinates of
   all the points in the picture, giving several pages of useless
   output.  
 \item[(d)] It is a very interesting and important fact that the $n$'th
  prime is approximately $n(\ln(\ln(n))+\ln(n)-1)$.  To check this,
  plot the function $x(\ln(\ln(x))+\ln(x)-1)$ for $x$ from $1$ to
  $100$, then enter \verb~pic2:=%:~
  (with a colon, again) to save the picture.  Then enter
  \verb~display(pic1,pic2);~ to compare the two plots. 
 \item[(e)] Repeat for the first $1000$ primes.  This can be done in
  one step, as follows:
\begin{verbatim}
   display(
    listplot([seq(ithprime(i),i=1..1000)],style=POINT,symbol=POINT),
    plot(x*(ln(ln(x)) + ln(x) - 1),x=1..1000)
   );
\end{verbatim}
  (To lay this out neatly, as above, hold the SHIFT key when you press
  RETURN at the end of each line.  Maple will not attempt to carry out
  the command until you press RETURN without the SHIFT key.) 
 \end{itemize}
\end{exercise}

\begin{exercise}
 \begin{itemize}
  \item[(a)] Enter \verb~20!;~ to calculate the number
   $20!=1\tm 2\tm 3\tm\dotsb\tm 19\tm 20$.  Then enter
   \verb~evalf(20!);~ to give the same answer in scientific notation,
   making it easier to see the approximate size: about
   $0.24\tm 10^{19}$. 
  \item[(b)] Now calculate $1!,2!,\dotsc,20!$ using the
   \verb~seq~ command.  Then plot these values, as
   in Exercise~\ref{ex-plotprimes}(c). 
   The result is not very informative --- why not? 
  \item[(d)] Instead, plot the values of $\ln(n!)$ for $n$ from $1$ to
   $20$; this gives a more useful picture.  Save the graph by entering
   \verb~pic1:=%:~.  
  \item[(e)] Define $f(x)=\sqrt{2\pi} x^{x+1/2} e^{-x}$
   (using \note{\NOTEarrow}).  It is an
   interesting and useful fact that $f(n)$ is a good approximation to
   $n!$, and so $\ln(f(n))$ is a good approximation to $\ln(n!)$.  To
   check this, plot $\ln(f(x))$ for $x$ from $1$ to $20$ (using the
   ordinary \verb~plot()~ command rather than \verb~listplot()~). 
   Save the result by entering \verb~pic2:=%:~,
   then enter \verb~display(pic1,pic2);~ to combine the two graphs. 
 \end{itemize}
\end{exercise}

\begin{exercise}
 Define a sequence of functions $f(n,c)$ by $f(0,c)=0.5$ and
 $f(n+1,c)=c\,f(n,c)(1-f(n,c))$, so for example
 \begin{align*}
  f(1,c) &= c\times 0.5 \times (1-0.5) = 0.25 c \\
  f(2,c) &= c \times (0.25 c) * (1-0.25c) = 0.25c^2 - 0.0625 c^3. 
 \end{align*}
 (This comes from a very simple model of population dynamics, where
 $c$ is a parameter depending on the reproductive behaviour of a
 certain species, and $f_n(c)$ is the population density of that
 species in the $n$'th year.)  It turns out that the long-term
 pattern of this sequence depends in an intricate and interesting way
 on $c$; this is the starting point of the theory of chaotic
 dynamics. 

 We can enter the definition in Maple as follows:
\begin{verbatim}
 f := proc(n,c)
  option remember;
  if n = 0 then
   return 0.5;
  else 
   return c * f(n-1,c) * (1 - f(n-1,c));
  end if;
 end;
\end{verbatim}
 Now use the \verb~listplot~ command to plot the values $f(i,2.9)$
 for $i$ from $1$ to $500$.  Just plot the points, not the lines
 joining them \note{\NOTElistplotstyle}.  You should see that the
 points bounce around a bit when $i$ is small, but they settle down so
 that $f(i,2.9)$ is close to $0.655$ when $i$ is large.  Now change
 the parameter $c$ from $2.9$ to $3.1$.  How does the pattern change?
 Try various values between $2.9$ and $3.1$ to see in more detail what
 happens and when.  Then look at the range $3.4\leq c\leq 3.5$, then
 $3.52\leq c\leq 3.56$, then $3.62\leq c\leq 3.63$. 

\end{exercise}

\chapter{Differentiation 1}

\begin{exercise}\label{ex-Li}
 On the course home page (\verb~www.shef.ac.uk/nps/MAS100~), click on
 the link marked \verb~defineq.mws~.  This should open up a worksheet 
 containing a single line starting
 \verb~q:=sscanf("f*6...~.  Click on this line and press
 ENTER.  This will set up a ``mystery function'' called
 $q(x)$, whose properties we will investigate. 
 \footnote{For a non-mathematical challenge, you can enter
  \verb~?option~ and \verb~?interface~, read the resulting help pages
  carefully, and work out how to print out a more
  comprehensible definition of $q(x)$.  You will then need
  to read more help pages to understand the definition. 
 }
 \begin{itemize}
  \item[(a)] Plot $q(x)$.  Experiment to find a range of $x$ values
   that shows all the main features of the graph. 
  \item[(b)] For which values of $x$ do we have $q'(x)<0$?  (You can
   answer this by just looking at the graph.)  How about $q'(x)=0$ or
   $q'(x)>0$? 
  \item[(c)] Enter \verb~Q:=(x,h)->(q(x+h)-q(x))/h;~, so that $q'(x)$
   is the limit of $Q(x,h)$ as $h$ tends to zero
   \note{\NOTEmultifunc}.  Put \verb~e:=exp(1);~ for
   convenience \note{\NOTEexp}.  Tell Maple to do all numerical
   calculations to 30 digits \note{\NOTEdigits}. 
  \item[(d)] Enter \verb~Q(e^2,0.1)~ to get an approximate value for
   $q'(e^2)$.  Replace $0.1$ by a smaller number to get a better
   approximation.  What do you think is the exact value of $q'(e^2)$? 
  \item[(e)] Work out $q'(e^{-2})$ and $q'(e^4)$ in the same way. 
   Guess the formula for $q'(e^t)$. 
  \item[(f)] Suppose that $x>0$.  What is the number $t$
   such that $x=e^t$?  Deduce a formula for $q'(x)$ (valid for $x>0$). 
 \end{itemize}
\end{exercise}


\begin{exercise}\label{ex-rolle}
 We will call the following \emph{Rolle's principle}: between any two
 roots of a function $f(x)$, there is at least one root of $f'(x)$. 
 \footnote{ As we will see, there are some exceptions.  However, there
   is a more precisely formulated version, called \emph{Rolle's
   Theorem} , for which there are no exceptions (that is what
   ``theorem'' means).  
 } 
 \begin{itemize}
  \item[(a)] How can you find the roots of $f'(x)$ by looking at the
   graph of $f(x)$? 
  \item[(b)] Define $f(x)=x^3-x$ using \note{\NOTEarrow}. 
   Plot the graph.  You should see that Rolle's principle is
   correct in this case.  We will now analyse this more precisely using
   Maple (although it would be easy to do it by hand). 
   \begin{itemize}
    \item Ask Maple \note{\NOTEroots} to find the roots of $f(x)$. 
    \item Ask Maple \note{\NOTEdiff,\NOTEleibniz} to find
     $f'(x)$.  (Note that \verb~f'(x)~ is \textbf{not} valid
     Maple syntax.) 
    \item Ask Maple to find the roots of $f'(x)$. 
    \item Plot $f(x)$ and $f'(x)$ together, and observe how the roots
     of $f(x)$ alternate with the roots of $f'(x)$. 
   \end{itemize}
  \item[(c)] Repeat part~(b) for the function
   $g(x)=\sin(x)+\sin(3x)/3$.  Maple will give you six roots for
   $g(x)$, but four of them are complex numbers, so they can be
   ignored.  It will just give three real roots for $g'(x)$.  However,
   both $g(x)$ and $g'(x)$ are periodic, so they really have
   infinitely many roots.  You should work out from the graph
   where the rest of the roots lie.  (You could ask Maple to do this
   by setting \verb~_EnvAllSolutions:=true;~, but it does not do a
   good job: the answer comes out in a complicated and confusing
   form.) 
  \item[(d)] Can you explain roughly why Rolle's principle is true? 
   (Part~(a) is relevant.) 
  \item[(e)] Consider $h(x)=\tan(x)$.  Is Rolle's principle true in
   this case?  How does this relate to your answer to~(d)?  
 \end{itemize}
\end{exercise}

\begin{exercise}\label{ex-schwartzian}
 Suppose that $y$ is a function of $x$, and we write $y'=dy/dx$ and so
 on.  The \emph{schwartzian derivative} of $y$ is defined to be
 \[ S(y) = y'''/y' - \frac{3}{2}(y''/y')^2. \]
 \begin{itemize}
  \item[(a)] Work out $S(x^n)$ by hand.  For which $n$ is $S(x^n)=0$? 
   For which $n$ is $S(x^n)$ undefined? 
  \item[(b)] Enter the definition into Maple:
\begin{verbatim}
  S := (u) -> diff(u,x,x,x)/diff(u,x) - (3/2)*(diff(u,x,x)/diff(u,x))^2;
\end{verbatim}
   Now use Maple to check your answer to~(a). 
  \item[(c)] Define $y=(ax+b)/(cx+d)$ (where $a$, $b$, $c$ and $d$ are
   constants), then ask Maple to calculate and simplify $y'$, $y''$,
   $y'''$ and $S(y)$.  You should find that $S(y)=0$.  In fact,
   functions of this form are the \emph{only} ones for which $S(y)=0$. 
  \item[(d)] Now define $z=(ap^x+b)/(cp^x+d)$ (where $p$ is yet
   another constant) and simplify $S(z)$.  You should find that $S(z)$
   is a constant, not depending on $x$.  However, some creativity is
   required to persuade Maple to do the right simplifications.  Here
   is one approach that works: expand out $1/S(z)$, simplify it, and
   then take $1$ over the result. 
  \item[(e)] Now define 
   \[ T(u) = (u')^{1/2} ((u')^{-1/2})''. \] Enter this
   definition in Maple (with syntax similar to part~(b)). 
   Choose some functions $w$ at random (eg $w=\ln(x)$, or
   $w=\sin(x)^2$) and calculate $S(w)$ and $T(w)$.  What do
   you notice?  Can you prove it? 
 \end{itemize}
\end{exercise}

\begin{exercise}
 Consider the function $y=\cos(-2\ln(x))$ (defined for $x>0$).
 \begin{itemize}
  \item[(a)] Plot the graph from $x=0$ to $x=0.01$, and then from
   $x=0$ to $x=1000$.
  \item[(b)] One of the roots of $y$ is at $x=a_0=e^{-\pi/4}$.  Let
   $m_0$ be the gradient $dy/dx$ at $x=a_0$.  Find and simplify
   $m_0$.  
  \item[(c)] What is the equation of the line $L_0$ that passes
   through $(a_0,0)$ with gradient $m_0$?  Plot this line together
   with the graph of $y$, for $x$ from $0$ to $2a_0$.
  \item[(c)] What are the other roots of $y$?  Repeat part~(b) for
   several other roots.  What do you notice about the lines $L_i$?
   Can you prove it?
 \end{itemize}
\end{exercise}

\chapter{Differentiation 2}

\section{Maxima and minima}

\begin{exercise}\label{ex-polyexp-max}
 Plot the function $y=x^ne^{-x}$ for various values of $n$.
 Solve $dy/dx=0$ and so find the maximum value of $y$
 (assuming $x>0$).
\end{exercise}

\begin{exercise}\label{ex-poly-max}
 Consider the function
 \[ p(x) = -10x^6+156x^5-945x^4+2780x^3-4080x^2+2880x. \]
 Plot the graph for a suitable range.  Find all the critical
 points of $p(x)$ (where $p'(x)=0$).  Which of these are
 inflection points (where $p''(x)=0$ as well)?  What is the
 maximum value of $p(x)$, and for what value of $x$ does it
 occur? 
\end{exercise}

\section{Implicit derivatives}

\begin{exercise}\label{ex-fermat-spiral}
 Put $u=x\sin(x^2+y^2)-y\cos(x^2+y^2)$.  Plot the curve
 where $u=0$, using the \verb+implicitplot+ command
 \note{\NOTEimplicitplot} .  Remember that you need
 \verb~with(plots):~ to make this work
 \note{\NOTEwithplots}.  A reasonable range is to let $x$
 and $y$ run from $-5$ to $5$, and you should ask Maple to
 plot extra points \note{\NOTEgrid} to get a decent picture.
 The curve is called a \emph{Fermat spiral}.

 Now find the slope of the curve using the
 \verb+implicitdiff+ command \note{\NOTEimplicitdiff}, and
 call the answer \verb+slope1+.  Maple will give the answer
 in a form involving $\sin(x^2+y^2)$ and $\cos(x^2+y^2)$.
 Can you use the relation $u=0$ to rewrite \verb+slope1+ in
 a form that does not involve $\sin$ or $\cos$?

 We next claim that the curve can also be described
 parametrically by $(x,y)=(t\cos(t^2),t\sin(t^2))$.  In
 checking this, we will use the symbol \verb~xt~ for ``$x$
 in terms of $t$'', and thus enter
\begin{verbatim}
  xt := t*cos(t^2);  yt := t*sin(t^2);
\end{verbatim}
 Use the \verb~subs~ command \note{\NOTEsubs} to substitute
 \verb~xt~ for \verb~x~ and \verb~yt~ for \verb~y~ in $u$,
 and simplify the result.  You should get zero, indicating
 that the point $(t\cos(t^2),t\sin(t^2))$ really does lie on
 the curve $u=0$.  (It is not very hard to do this by hand
 as well.)  You can also enter \verb~plot([xt,yt,t=-5..5]);~
 and see that you get the same picture as before. 

 Now calculate $dx/dt$ and $dy/dt$, and so get a formula for
 $dy/dx$ in terms of $t$.  Call this one \verb~slope2~.  Use
 the \verb~subs~ command to rewrite \verb~slope1~ in terms
 of $t$ (rather than $x$ and $y$), simplify the result, and
 check that it is the same as \verb~slope2~. 
\end{exercise}

\begin{exercise}\label{ex-trefoil}
 Consider the curve with equation $u=0$, where
 \[ u = (x^2+y^2)^2 + 85(x^2+y^2) - 500 + 18x(3y^2-x^2). \]
 It turns out that this can also be given parametrically by 
 \[ (x,y) =
     (6\cos(t) + 8\cos(t)^2 - 4, 2\sin(t)(3-4\cos(t))). 
 \]
 Analyse this situation just as in the previous question:
 plot the graph in two different ways, find the slope in two
 different ways, and check that they are really the same. 
\end{exercise}

\section{Higher derivatives}

\begin{exercise}\label{ex-Psi}
 Enter the following definition into Maple
 \note{\NOTEseriesdef,\NOTEmultidiff}:
 \[ r(n) =
    \frac{d^n}{dx^n}\left(\frac{x^n\ln(x)}{n!}\right). 
 \]
 Find $r(n)$ for a reasonable range of numbers $n$, using
 the \verb~seq(...)~ command \note{\NOTEseq}.  Then
 experiment to find a formula for $r(n)-r(n-1)$.  Finally,
 deduce a formula for $r(n)$, of the form
 \[ r(n) = \text{something} +
            \sum_{k=1}^{n}\text{(something)}. 
 \]
\end{exercise}

\begin{exercise}\label{ex-diffop-jordan}
 Put $y=t^2e^t$.  Simplify the expression 
 \[ z = y''' + a y'' + b y' + c y, \]
 where $a$, $b$ and $c$ are constants.  Hence find $a$, $b$
 and $c$ such that $y$ satisfies the differential equation 
 \[ y''' + a y'' + by' + c y = 0 \]
 for all $t$. 
\end{exercise}

\begin{exercise}\label{ex-hermite}
 Put $p(n)=e^{x^2}\frac{d^n}{dx^n}(e^{-x^2})$.  Enter this
 definition in Maple \note{\NOTEseriesdef,\NOTEmultidiff}, and then
 calculate $p(n)$ for $n$ from  $1$ to $10$.  (It is best to
 define $p(n)$ to be \verb~sort(expand(exp(x^2) * ...))~, to
 get the  answer in a convenient form.  To print
 $p(1),\ldots,p(10)$ on separate lines, enter
 \verb~seq(print(p(n)),n=1..10)~ rather than just
 \verb~seq(p(n),n=1..10)~.)  
 
 Write down as many things as you can about the general form
 of $p(n)$, distinguishing between the case where $n$ is
 even and the case where $n$ is odd.  Using these, predict
 as much as you can about $p(12)$, and then
 check your predictions. 
 
 In doing this question, you should see the numbers $2$,
 $4$, $8$, $16$, $32$, $64$ and so on, which you should
 recognise as powers of $2$.  You will also see the numbers
 $2$, $12$, $120$, $1680$, $30240$, which are much less
 likely to be familiar.  To see what they are, enter the
 sequence in the search box at
 \verb+http://www.research.att.com/~njas/sequences/+.  (The
 response is long and complicated, but the formula that you
 need appears in the first few lines.) 

 It turns out that $p(n)$ is closely related to the Hermite
 polynomials, which Maple calls \verb~HermiteH(n,x)~.  Enter 
 \verb~q:=(n)->simplify(HermiteH(n,x));~, then work out
 $q(n)$ for various values of $n$, then write down the
 precise relationship between $p(n)$ and $q(n)$.  
\end{exercise}

\chapter{Integration 1}

\section*{Setup}

Visit \verb~http://www.shef.ac.uk/nps/MAS100~ and click with your
\textbf{right} mouse button on the link marked \verb~ratint.mws~.
Then select \verb~Save Target As..~, save the worksheet somewhere,
start up Maple separately, and then open the worksheet using the File
menu.  When the worksheet has started up, click the button on the
Maple toolbar marked with a triple exclamation mark (!!!).  This will
execute all the commands in the worksheet, which does two things:
\begin{itemize}
 \item[(a)] It defines a command called \verb~ratint()~.
  This works in essentially the same way as the usual
  integration command \verb~int()~, but it gives the answer
  in a form which is slightly better for the questions on
  this sheet.
 \item[(b)] It defines a number of rational functions
  $g_1(x),g_2(x),\dotsc$ (which you should enter as
  \verb~g[1](x)~, \verb~g[2](x)~ and so on).  These can be
  used as examples later (when the question says ``for
  various rational functions \ldots'').
\end{itemize}
If you use \verb~restart~ in doing this problem sheet, you
will lose the definition of \verb~ratint()~ and the
functions $g_i(x)$.  It is better to use \verb~unassign~
instead.  Remember that the syntax is like
\verb~unassign('a','b')~, with single quote marks.

\section*{Questions}

Your task is to answer the following questions, using a
mixture of mathematical thinking and experimental
calculations with Maple.  When calculating integrals, use
\verb~ratint(...,x);~ instead of \verb~int(...,x);~.

\begin{exercise}
 When $x$ is large, the graph of
 $y=\displaystyle\int\left(\frac{x^3+1}{x^2+1}\right)^3\,dx$
 looks like $y=cx^n$ for some constant $c$ and integer $n$.
 Find $c$ and $n$.
\end{exercise}

\begin{exercise}
 Let $a$, $b$, $c$ and $d$ be nonzero constants.  The graph
 of $\displaystyle y=\int\frac{ax^2+b}{cx^2+d}\,dx$ looks
 like a straight line for large $x$.  What is the slope of
 the line?
\end{exercise}

\begin{exercise}
 Are the following true or false?
 \begin{itemize}
  \item[(a)] Integrals of rational functions sometimes
   involve terms like $a\ln(x^2+ux+v)$, where $a$, $u$ and
   $v$ are constants.
  \item[(b)] Integrals of rational functions sometimes
   involve terms like $x\ln(x+u)$, where $u$ is constant.
  \item[(c)] Integrals of rational functions sometimes
   involve terms like $a\ln(x+u)^2$, where $a$ and $u$ are
   constants. 
 \end{itemize}
\end{exercise}

\begin{exercise}
 For various different values of $b$ and $c$ (which may be
 positive, negative or zero), do the following:
 \begin{itemize}
  \item[(i)] Plot $y=x^2+bx+c$.
  \item[(ii)] Calculate $b^2-4c$.
  \item[(iii)] Find $\displaystyle\int\frac{dx}{x^2+bx+c}$,
   and observe whether it involves the functions $\arctan$
   or $\ln$. 
 \end{itemize}
 What is the relationship between~(i), (ii) and (iii)?
\end{exercise}

\begin{exercise}
 For various rational functions $g(x)$, do the following:
 \begin{itemize}
  \item[(i)] Find $\int g(x)\,dx$, and look for any terms
   like $\ln(|x-u|)$ (but ignore terms like $\ln(x^2+px+q)$).
  \item[(ii)] Plot $g(x)$ together with $\int g(x)\,dx$.
   (You will generally need to specify the vertical range in
   order to get a useful picture.)
 \end{itemize}
 How are the numbers in~(i) related to the pictures in~(ii)?  
\end{exercise}

\begin{exercise}
 For various rational functions $g(x)$, do the following:
 \begin{itemize}
  \item[(i)] Enter \verb~d:=denom(factor(g(x)));~ to find
   the denominator of $g(x)$ (in other words, the term on
   the bottom when $g(x)$ is written as a single fraction).
  \item[(ii)] Plot $d$.
  \item[(iii)] Find $\int g(x)\,dx$, and look for terms like
   $a/(x-u)$ or $a/(x-u)^n$. 
 \end{itemize}
 How are the numbers $u$ and $n$ in~(iii) related to the
 answers to~(i) and~(ii)?
 
 Now enter \verb~r:=randpoly(x)/randpoly(x);~ to generate a
 random rational function, and then \verb~ratint(r,x);~ to
 integrate it.  Now go back in the worksheet to the line
 \verb~r:=randpoly(x)/randpoly(x);~ and press ENTER twice,
 to generate and integrate a new example.  Repeat this many
 times.  You should see that you only get $\ln()$ and
 $\arctan()$ terms, together with multiples of $x^n$ for
 some small values of $n$.  Can you see why terms like
 $a/(x-u)$ hardly ever occur?
\end{exercise}


\chapter{Integration 2}

This sheet deliberately gives very little help with Maple syntax, and
the same will be true for the remaining sheets.  As you will see from
the past papers on the course website, you will need a fairly detailed
knowledge of the syntax for the exam.

\begin{exercise}\label{ex-legendre}
 Enter the following definitions:
 \[
  p_0 = \sqrt{\tfrac{1}{2}}            \hspace{2em}
  p_1 = \sqrt{\tfrac{3}{2}} x          \hspace{2em}
  p_2 = \sqrt{\tfrac{5}{8}}(3x^2 - 1)  \hspace{2em}
  p_3 = \sqrt{\tfrac{7}{8}}(5x^3 - 3x)
 \]
 (Remember that $p_2$ should be entered as
 \verb~p[2]~, and so on.)  Find
 $\int_{-1}^1p_ip_j\,dx$ for various $i$ and $j$
 (including the case $i=j$).  What pattern do you observe?
 There is a function $p_4$ of the form $ax^4+bx^2+c$
 (with $a>0$) that makes the pattern continue.  What are
 $a$, $b$ and $c$?  (The first step is to define
 $p_4=ax^4+bx^2+c$, leaving $a$, $b$ and $c$ as
 variables.  Then calculate $\int_{-1}^1p_2p_4\, dx$
 and so on, giving answers that depend on $a$, $b$ and $c$.
 This should give you some equations relating $a$, $b$ and
 $c$, which you should solve.  If you use the \verb~solve~
 command, you may find it helps to set
 \verb~_EnvExplicit:=true~ first.)

 Now enter the definition
 \[ q(n) =
     \frac{\sqrt{n+1/2}}{2^n\,(n!)}
      \frac{d^n}{dx^n}\left((x^2-1)^n\right).
 \]
 Check that $q(n)=p_n$ for $n=1,\ldots,4$.
\end{exercise}

\begin{exercise}
 For various positive integers $n$ and $m$, calculate and
 factorise the integral $\int x^n\ln(x)^m\,dx$.  What terms
 appear in the answer?  How does this depend on $n$ and $m$?
 Write down your conclusions as a self-contained statement
 that would make sense to someone who had not read the
 question. 
\end{exercise}

\begin{exercise}
 You should be aware that there are many integrals for which
 the answer simply cannot be written in terms of familiar
 functions.  In same cases, Maple will write the answer in
 terms of more obscure functions instead.  For example, try
 the following:
 \[ \text{\bf (a)} \int e^{-x^2}\,dx \hspace{2em}
    \text{\bf (b)} \int \frac{dx}{\ln(x)} \hspace{2em}
    \text{\bf (c)} \int \frac{dx}{\sqrt{1-x^2}\sqrt{1-2x^2}} \hspace{2em}
    \text{\bf (d)} \int (x^8+1)^{-1/2}\,dx
 \]
 In some other cases, Maple will just give up.  For example,
 try the following:
 \[ \text{\bf (e)} \int \sin(x)\ln(\ln(x))\,dx \hspace{2em}
    \text{\bf (f)} \int \sin(\sin(\sin(x)))\,dx \hspace{2em}
    \text{\bf (g)} \int \frac{dx}{\sqrt{1+x+x^{10}}}
 \]
 What is the simplest function that you can find that makes
 Maple give up?  (There is one that you can enter with just
 three characters.)
\end{exercise}

\begin{exercise}
 Consider the function
 \[ y = \sin(x) + \tfrac{1}{3}\sin(3x) +
        \tfrac{1}{5}\sin(5x) + \tfrac{1}{7}\sin(7x).
 \]
 Plot $y$ and $\int y\,dx$.  Describe and explain the
 relationship between the shapes of these two graphs.
\end{exercise}

\begin{exercise}
 Put $y=x\,e^{-x}\sin(20x)$ and $z=20 \int y\,dx$.  Plot $y$ and
 $z$ for various ranges of $x$.  Describe in detail, and
 explain, the relationship between the two graphs.  Then
 plot $y^2+z^2$.  This is very close to a function with a
 much simpler formula; work out what it is.
\end{exercise}

\begin{exercise}
 Find $a$, $b$ and $c$ such that
 $\int_0^\infty x^k(ax^2+bx+c)e^{-x}\,dx=k$ for $k=1,2$ and $3$.
\end{exercise}

\begin{center}
 {\huge Maple syntax revision}
\end{center}

\begin{exercise}
 Find an approximate solution to
 $x(\ln(\ln(x))+\ln(x)-1)=541$ close to $x=100$. 
\end{exercise}

\begin{exercise}
 What would you enter to generate the sequence
 $a+b^2,a+b^3,a+b^4,\dotsc,a+b^{10}$? 
\end{exercise}

% \begin{exercise}
%  Find the Taylor series of $\ln(\ln(x))$ around $x=e$, to
%  order $4$.  (If your answer involves $\ln(e)$ then you have
%  made a syntactic mistake, because $\ln(e)=1$ and Maple
%  should know this.  Remember that $e$ cannot simply be
%  entered as \verb~e~.)
% \end{exercise}

\begin{exercise}
 Plot the curve $x^4+y^4=1$ together with the curve given
 parametrically by $x=\cos(t)/(2+\cos(4t))$ and
 $y=\sin(t)/(2+\cos(4t))$. 
\end{exercise}

\begin{exercise}
 Find $\pi^{76}/e^{87}$ to 100 decimal places.
\end{exercise}

\begin{exercise}
 Simplify the expression
 $\displaystyle \sin(2x)\tan(2x)\frac{d^2}{dx^2}\log(\tan(x))$.
\end{exercise}


\chapter{Integration 3}

Please refer to the ``Notes on Maple syntax'' as you do
these questions.

\begin{exercise}
 Enter the definitions $x_1=\sin(\theta)\sin(\phi)$,
 $x_2=\sin(\theta)\cos(\phi)$, and $x_3=\cos(\theta)$.
 Simplify $x_1^2+x_2^2+x_3^2$.
\end{exercise}

\begin{exercise}
 Enter the definition
 \[ u = 8 - (x-y)^2 (x+y)^2 (16 - 4x^2 - y^4). \]
 Substitute $x=2\cos(t)$ and $y=2\sin(t)$ in $u$ and
 simplify the result.  Try both the \verb~simplify()~
 command and the \verb~combine()~ command.
\end{exercise}

\begin{exercise}
 Solve the equations
 \begin{align*}
  ax+by+cz &= 1 \\
  ay+bz+cx &= 1 \\
  az+bx+cy &= 1
 \end{align*}
 to find $x$, $y$ and $z$ in terms of $a$, $b$ and $c$.
\end{exercise}

\begin{exercise}
 Find an approximate solution of $x^4+\sin(x)=10^4$ close
 to $x=10$.  Arrange your syntax so that Maple gives you an
 equation, not just a number.
\end{exercise}

\begin{exercise}
 Enter the definition $f(x)=(e^x+x^7)/(e^x-x^7)$.  Find
 a numerical approximation to $f(5)$.  Then find numerical
 approximations for the whole sequence
 $f(0),f(1),\dotsc,f(50)$. 
\end{exercise}

\begin{exercise}
 Plot the curve $x^4+y^4=4$, together with the curve given
 by $x=(2+\sin(8t))\cos(t)$ and $y=(2+\sin(8t))\sin(t)$ (for
 $0\leq t\leq 2\pi$).
\end{exercise}

\begin{exercise}
 Plot $y=\tan(\pi x)$, $y=\cot(\pi x)$, $y=-1$ and $y=1$ on
 the same graph, for $x$ from $-4$ to $4$.  Tell Maple to
 restrict the vertical range from $-2$ to $2$, to use the
 same scale on the two axes, and to skip over
 discontinuities.
\end{exercise}

\begin{exercise}
 Enter the definitions
 \begin{align*}
  f(t) &= \frac{(2+\sqrt{3})t-1}{2+\sqrt{3}+t} \\
  g(t) &= f(f(f(t))) \\
  h(t) &= g(g(t)).
 \end{align*}
 Simplify $g(t)$ and $h(t)$.
\end{exercise}

\begin{exercise}
 Find and simplify $\displaystyle
 \left(\frac{1}{y}\frac{d^8y}{dx^8}\right)^{1/8}$, where
 $y=\sin(10x)$. 
\end{exercise}

% \begin{exercise}
%  Find the Taylor series of $a\sin(x)+b\sin(2x)+c\sin(3x)$
%  about $x=0$, to order $7$.  Convert the result to an
%  ordinary polynomial, with no $O(x^7)$ term.  Now find $a$,
%  $b$ and $c$ such that
%  $a\sin(x)+b\sin(2x)+c\sin(3x)=30x+O(x^7)$.
% \end{exercise}

\begin{exercise}
 Plot the curves $y=n^{3/2}x^n(1-x)^2$ for $n=1,\dotsc,10$ on the same
 graph, with $x$ running from $0$ to $1$.  Ideally you should use the
 \verb~seq()~ command, otherwise you will have to do a lot of typing.
\end{exercise}

\begin{exercise}
 Put $\displaystyle y=\frac{x^4(1-x)^4}{1+x^2}$.  Find the
 indefinite integral $\int y\,dx$, the definite integral
 $\int_0^1 y\,dx$ and the derivative $dy/dx$.  Plot the
 graph of $y$ from $x=0$ to $x=1$; you should see a single
 hump.  Find numerical approximations to the position and
 height of the hump.
\end{exercise}

\begin{exercise}
 Define $a(n)$ to be the approximate numerical value of
 $(1+1/n)^n/e$.  Find the sequence
 $a(1),a(2),\dotsc,a(100)$.  What is the first $n$ for which
 $a(n)>0.99$? 
\end{exercise}

\chapter{Taylor series}

\section{Taylor series}\label{sec-taylor}

\begin{exercise}\label{ex-ln}
 We will study the Taylor series at $x=0$ of the function 
 \[ y = \ln\left(\sqrt{\frac{1+x}{1-x}}\right).  \]
 This has the form $y=\sum_{k=0}^\infty a_kx^k$, where $a_k$
 is $1/k!$ times the value of $d^ky/dx^k$ at $x=0$.  You
 should start by entering the definition of $y$.
 \begin{itemize}
  \item[(a)] Find \note{\NOTEmultidiff} and simplify
   $d^5y/dx^5$.  Then put $x=0$ using the \verb~subs~
   command \note{\NOTEsubs}, and divide by $5!$ to get $a_5$.
  \item[(b)] Find $a_1$, $a_2$, $a_3$ and $a_4$ in the
   same way.
  \item[(c)] To do this more efficiently, enter 
\begin{verbatim}
 a := (n) -> subs(x=0,diff(y,x$n))/n!;
\end{verbatim}
   then use the \verb~seq~ command \note{\NOTEseq} to
   calculate $a(1),\dotsc,a(5)$ in one go.  (This syntax does not
   work properly when $n=0$, but it is easy to see that
   $a_0=0$ anyway.)
  \item[(d)] Now enter \verb~add(a(k)*x^k,k=1..12);~ to get
   the $13$'th order Taylor series for $y$ at $x=0$.  Then do the
   same thing more easily using the \verb~series~ command
   \note{\NOTEtaylor}. 
  \item[(e)] Guess the complete Taylor series for $y$ at $x=0$.
 \end{itemize}
\end{exercise}

\begin{exercise}\label{ex-sin}
 \begin{itemize}
  \item[(a)] Find the $12$th order Taylor series for
   $\sin(x)$ at $x=0$, using the \verb~series~ command
   \note{\NOTEtaylor}.  Convert the answer to an ordinary
   polynomial as in \note{\NOTEtaylorconv}. Call the result $s$. 
  \item[(b)] Plot $s$ and $\sin(x)$ together
   \note{\NOTEtwoplots} for $x$ from
   $-8$ to $8$, with the vertical range also restricted
   \note{\NOTEvertrange}  from $-10$ to $10$.  You should see that
   the two graphs are very close together for $x$ between
   $-5$ and $5$, but that they diverge very rapidly outside
   that range. 
  \item[(c)] Now define $t(n)$ to be the $n$'th order Taylor
   series for the function $\sin(x)$ at $x=0$.  (Use syntax like 
   \verb~t:=(n)->convert(series(...)...)~).  Plot $t(n)$ and
   $\sin(x)$ together (with the same ranges as in~(b)) for
   various $n$.  As you make $n$ bigger, the two graphs will
   get closer together.  How big must $n$ be for the two
   graphs to look identical?  What then happens if you
   expand the horizontal range to run from $-10$ to $10$? 
  \item[(d)] To plot $\sin(x)$ together with
   $t(2),t(6),t(10),t(14),\ldots,t(42)$, enter
\begin{verbatim}
   plot([sin(x),seq(t(4*n+2),n=0..10)],x=0..20,-2..10);
\end{verbatim}
   Now modify this to get a good picture of
   $t(4),t(8),\ldots,t(44)$.  Why do we not bother with
   $t(n)$ for odd $n$? 
  \item[(e)] Now define $r(n)$ to be the $n$'th order Taylor
   series for $\cos(x)$ at $x=0$.  Expand out
   $t(10)^2+r(10)^2$, and call the result $q$.  What should
   this be equal to, approximately?  How could you check this? 
 \end{itemize}
\end{exercise}

\begin{exercise}\label{ex-rat}
 Find the Taylor series for $y=\frac{x(1+x)}{(1-x)^3}$ at
 $x=0$, to some reasonably high order.  You should be able
 to guess from this that
 \[ y = \sum_{k=1}^\infty \text{(something)} x^k. \]
 You can ask Maple to confirm this by entering
\begin{verbatim}
  sum((something) * x^k,k=1..infinity);
\end{verbatim}
 You should see that Maple's answer is the same as $y$
 (rearranged slightly). 
\end{exercise}

\begin{exercise}\label{ex-sin-shift}
 Enter \verb~series(sin(x),x=Pi/2,12)~ to find the 12th
 order Taylor series of $\sin(x)$ around $x=\pi/2$.  How
 is this related to the 12th order Taylor series of
 $\cos(x)$ around $x=0$?  Why? 
\end{exercise}

\section{Maple syntax revision}

The questions in this section all cover things you have done
before, but this time there are no hints about the syntax. 
Look at the notes on Maple
(\verb~http://www.shef.ac.uk/nps/MAS100/notes/primer.pdf~)
and/or the earlier lab sheets if there are things that you
do not remember. 

\begin{exercise}
 Find $\cos(\ln(\pi+20))$ to $20$ decimal places. 
\end{exercise}

\begin{exercise}
 Enter the definition $f(x)=x^2-29/16$.  Ask Maple to work
 out $f(0)$.  If it just gives you ``$f(0)$'' back, then you
 used the wrong syntax to define $f$; read Section~10 of the
 Maple notes.  What is $f(f(f(-1/4)))$? 
\end{exercise}

\begin{exercise}
 Using the various commands for manipulating
 algebraic expressions, find out the relationship between
 \[ a = (1+x)^5-3(1+x)^4+5(1+x)^3-3(1+x)^2+3(1+x)+3. \]
 and
 \[ b = (7x^2-6x-x^8)/(x-1)^2. \]
\end{exercise}

\begin{exercise}
 Find the coefficient of $x^6$ in the expression
 \[
  \left(\frac{(x^{12}-1)(x^2-1)}{(x^6-1)(x^4-1)}\right)^{10}. 
 \]
 (You will need to simplify the expression first.) 
\end{exercise}

\begin{exercise}
 The equations
 \begin{align*}
  x^2+y^2+z^2 &= 9 \\
  (x-1)^2 + (y-1)^2 + (z-1)^2 &= 2 \\
  4x^2+yz &= 2x(y+z) 
 \end{align*}
 have only one solution in which $x$, $y$ and $z$ are all
 integers.  What is that solution? 
\end{exercise}

\begin{exercise}
 Find all the (infinitely many) solutions to the equation
 $\sin(\theta)^2=3\cos(\theta)^2$. 
\end{exercise}

\begin{exercise}
 Find the solution to the equations $x^2-y^2=2xy=1$ for
 which $x$ and $y$ are real numbers and $x>0$.  Your answer
 should not involve the word \verb~RootOf~. 
\end{exercise}

\begin{exercise}
 Find an approximate solution to $x=\log(x+20)$ close to
 $x=3$.  
\end{exercise}

\chapter{Revision} 

\section{Miscellaneous problems}

\begin{exercise}
 Plot the curve given by $x=\cos(t)/2-\cos(2t)/4$ and
 $y=\sin(t)/2-\sin(2t)/4$, together with the circle of radius $1/4$
 centred at $(-1,0)$, and two circles of radius $0.0945$ centred at
 the points $(-0.1225,\pm 0.7449)$.  Display all these in the same
 picture with the same scale in both directions, and with no axes.
 (This is a basic outline of the Mandelbrot set.  Google will find you
 much prettier pictures.)
\end{exercise}

\begin{exercise}
 Plot all the curves $|x|^n+|y|^n=n^{n/2}$ in the same picture for
 $n=1,\dotsc,9$.  Use the \verb~seq()~ command to avoid excessive
 typing, make sure that you use a range that includes all nine curves,
 and add an option to make Maple draw a more accurate picture.
\end{exercise}

\begin{exercise}
 Use the \verb~listplot~ command to plot the values $50^n/n!$ for $n$
 from $1$ to $100$.  Mark these values as separate points, not
 connected by lines.
\end{exercise}

\begin{exercise}
 There is a certain function $\zt(z)$ defined for complex numbers $z$,
 called the Riemann zeta function.  It features in the most famous
 open problem in all of pure mathematics, called the Riemann
 Hypothesis: is it true that whenever $s$ and $t$ are positive real
 numbers with $\zt(s+it)=0$, we have $s=1/2$?  There is a great deal
 of evidence for this but no proof.  If true, it will have important
 consequences for the distribution of prime numbers (and perhaps
 therefore for cryptography).

 \begin{itemize}
  \item[(a)] Enter \verb~with(MTM)~ to load the definition of
   $\zt(z)$.
  \item[(b)] Enter \verb~f:=(t)->Re(zeta(1/2+I*t))~, to
   define $f(t)$ to be the real part of $\zt(1/2+it)$.  Define $g(t)$ to
   be the imaginary part in the same way.
  \item[(c)] Plot the graphs of $f(t)$ and $g(t)$ together.  Note that
   $\zt(1/2+it)$ is zero when both graphs cross the horizontal axis in
   the same place.  What is the smallest positive value of $t$ where
   this happens?
  \item[(d)] Now instead plot the curve given parametrically by
   $x=f(t)$ and $y=g(t)$ (so $(x,y)$ corresponds to $\zt(1/2+it)$ in
   the Argand diagram).  In this picture, zeros of the $\zt$-function
   correspond to the places where the curve passes through the
   origin.  You should see that there are many of them (if you use a
   suitable range for $t$).
  \item[(e)] Now change the $1/2$ to some other value and repeat
   steps~(c) and~(d).  The Riemann hypothesis predicts that you should
   not find any zeros.  Thus, the two graphs in~(c) will never crosss
   the horizontal axis at the same time, and the curve in~(d) will
   never pass through the origin.  Check this.
 \end{itemize}
\end{exercise}

\begin{exercise}
 Put $y=\exp(-1/x)$, for $x>0$.  Investigate the behaviour of $y$ and
 the derivatives $d^ky/dx^k$, by calculating formulae and plotting
 graphs.  
\end{exercise}

\begin{exercise}\label{ex-expand-symmetric}
 Use Maple to expand out the expression
 \[ (x+y+z)^3 - 3(x+y+z)(xy+yz+zx) + 3xyz. \]
\end{exercise}

\begin{exercise}\label{ex-expand-unshift}
 Use Maple to expand out the following expressions:
 \begin{align*}
  & (x+1)^3 - 3(x+1)^2 + 3(x+1) - 1 \\
  & (x+1)^4 - 4(x+1)^3 + 6(x+1)^2 - 4(x+1) + 1 \\
  & (x+1)^5 - 5(x+1)^4 + 10(x+1)^3 - 10(x+1)^2 + 5(x+1) - 1.
 \end{align*}
 What is the pattern?  Can you see a simple explanation?
\end{exercise}

\begin{exercise}\label{ex-oddone}
 Four of the following five expressions are the same; which is the odd
 one out?  (You may assume that all variables are positive, and that no
 divisions by zero occur, so it is valid to use
 \verb+simplify(...,symbolic);+)
 \[ 
  \frac{u^{-1}+v^{-1}}{x^{-1}+y^{-1}}            \hspace{2em}
  \frac{uxv^{-1} + x}{uxy^{-1} + u}              \hspace{2em}
  \frac{(x-y)(u/v-v/u)}{(u-v)(x/y-y/x)}          \hspace{2em}
  \frac{(x^2+y^2)(u/v-v/u)}{(u^2+v^2)(x/y-y/x)}  \hspace{2em}
  \frac{u^3xy+u^2vxy}{u^3vx+u^3vy}
 \]
\end{exercise}
% \begin{solution}
%  All the expressions simplify to 
%  \[ \frac{(v+u)xy}{uv(y+x)}, \]
%  except for D, which simplifies to 
%  \[ \frac{(x^2+y^2)(u^2-v^2)yx}{uv(u^2+v^2)(x^2-y^2)}. \]
%  (To do this by hand, you should note that $u/v-v/u=(u-v)(1+u/v)$.)
% \end{solution}

\begin{exercise}\label{ex-torus}
 Simplify the expressions $U$ and $V$, defined below:
 \[ 
  X = (2+\cos(a))\cos(b) \hspace{2em}
  Y = (2+\cos(a))\sin(b) \hspace{2em}
  Z = \sin(a)            \hspace{2em}
  U = (X^2+Y^2+Z^2-5)/4  \hspace{2em}
  V = X/\sqrt{X^2+Y^2}
 \]
\end{exercise}
% \begin{solution}
%  As $\cos(b)^2+\sin(b)^2=1$, we have
%  \begin{eqnarray*}
%    X^2 + Y^2
%     &=& (2+\cos(a))^2\cos(b)^2 + (2+\cos(a))^2\sin(b)^2 \\
%     &=& (2+\cos(a))^2(\cos(b)^2+\sin(b)^2) = (2+\cos(a))^2 \\
%     &=& 4 + 4\cos(a) + \cos(a)^2.
%  \end{eqnarray*}  
%  It follows that 
%  \begin{eqnarray*}
%    X^2+Y^2+Z^2-5 
%     &=& 4 + 4\cos(a) + \cos(a)^2 + \sin(a)^2 - 5 \\
%     &=& 4 + 4\cos(a) + 1 - 5 = 4\cos(a),
%  \end{eqnarray*}
%  so $U=(X^2+Y^2+Z^2-5)/4=\cos(a)$.  Next, we have seen that
%  $X^2+Y^2=(2+\cos(a))^2$, so $\sqrt{X^2+Y^2}=2+\cos(a)$.  Thus
%  \[ V = \frac{X}{\sqrt{X^2+Y^2}} = 
%      \frac{(2+\cos(a))\cos(b)}{2+\cos(a)} = \cos(b).
%  \]
%  (Implicitly, this uses the fact that $\cos(a)$ is always greater
%  than or equal to $-1$, so $2+\cos(a)$ is always positive, so
%  there is no sign ambiguity in taking the square root.)
% \end{solution}

\begin{exercise}\label{ex-stereo}
 Put
 \begin{eqnarray*}
   x &=& \frac{2u}{u^2+v^2+1} \\
   y &=& \frac{2v}{u^2+v^2+1} \\
   z &=& \frac{u^2+v^2-1}{u^2+v^2+1} \\
   r &=& \sqrt{x^2+y^2+z^2} \\
   w &=& x/(1-z).
 \end{eqnarray*}
 Simplify $r$ and $w$.
\end{exercise}
% \begin{solution}
%  It works out that $r=1$ and $w=u$.  Indeed:
%  \begin{eqnarray*}
%   x^2+y^2+z^2 
%    &=&
%    \frac{(2u)^2 + (2v)^2 + (u^2+v^2-1)^2}{(u^2+v^2+1)^2} \\
%    &=& 
%    \frac{4u^2+4v^2+u^4+v^4+1-2u^2-2v^2+2u^2v^2}{(u^2+v^2+1)} \\
%    &=& 
%    \frac{u^4+v^4+1+2u^2+2v^2+2u^2v^2}{(u^2+v^2+1)^2} \\
%    &=& 
%    \frac{(u^2+v^2+1)^2}{(u^2+v^2+1)^2} = 1.
%  \end{eqnarray*}
%  Taking square roots, we see that $r=1$.  Next:
%  \begin{eqnarray*}
%   1-z 
%    &=&
%     1 - \frac{u^2+v^2-1}{u^2+v^2+1} 
%     = \frac{(u^2+v^2+1) - (u^2+v^2-1)}{u^2+v^2+1} \\
%    &=& \frac{2}{u^2+v^2+1} \\
%   \frac{1}{1-z} &=& \frac{u^2+v^2+1}{2} \\
%   w &=& \frac{x}{1-z} = 
%       \frac{2u}{u^2+v^2+1} \frac{u^2+v^2+1}{2} \\ 
%    &=& u.
%  \end{eqnarray*}
% \end{solution}

\begin{exercise}\label{ex-truefalse}
 Which of the following are true?  (You may assume that all variables
 are positive, and that no divisions by zero occur, so it is valid to
 use \verb+simplify(...,symbolic);+)
 \[ \begin{array}{rlrl}
  \text{(A)} &
   (x^{ab}x^{bc}x^{ca})^{1/(abc)}=x^{1/a + 1/b + 1/c} &
  \text{(B)} &
   \frac{(x/y)/z}{x/(y/z)}=1 \\
  \text{(C)} & 
   (x+1)^6-6x(x+1)^4+9x^2(x+1)^2=x^6+1 &
  \text{(D)} &
   \frac{2}{x^3}+\frac{2}{x^9-x^3}=\frac{1}{x^3-1}+\frac{1}{x^3+1} \\
  \text{(E)} &
   \text{ If } z=\frac{px+qy}{p+q}, \text{ then }
      \frac{px+qy+rz}{p+q+r}=z. & & 
 \end{array} \]
\end{exercise}
% \begin{solution}
%  To check equation A by hand, note that
%  \begin{eqnarray*}
%   (x^{ab}x^{bc}x^{ca})^{\frac{1}{abc}} 
%    &=& (x^{ab+bc+ca})^{\frac{1}{abc}} \\
%    &=& x^{(ab+bc+ca)/(abc)} \\
%    &=& x^{1/a + 1/b + 1/c},
%  \end{eqnarray*}
%  so A is true.  You can also ask Maple to simplify both sides,
%  but you must remember to us the {\tt symbolic} option to make
%  it deal with powers in the expected way.
%  \par
%  To check B, note that
%  \[ \frac{(x/y)/z}{x/(y/z)} = 
%     \frac{x/(yz)}{(xz)/y} = 
%     \frac{x}{yz}  \frac{y}{xz} = \frac{xy}{xyz^2} =
%     \frac{1}{z^2}.
%  \]
%  Thus, B is false.
%  \par
%  For C, just expand out the left hand side, by hand or using
%  Maple, to get $x^6+2x^3+1$ (not just $x^6+1$).  Thus, C is 
%  also false.
%  \par
%  Equation D is true, because both sides simplify to
%  $2x^3/(x^6-1)$.
%  \par
%  For E, you should enter {\tt z:=(p*x+q*y)/(p+q);} and then 
%  {\tt simplify((p*x+q*y+r*z)/(p+q+r)-z);} to get 0, showing
%  that the claim is true.  To do this by hand:
%  \begin{eqnarray*}
%   px+qy+rz &=& px+qy+ r\frac{px+qy}{p+q} \\
%    &=& \frac{1}{p+q}((p+q)px + (p+q)qy + rpx + rqy) \\
%    &=& \frac{1}{p+q}(p^2x+pqx+prx + pqy+q^2y+rqy) \\
%    &=& \frac{1}{p+q}(p+q+r)(px+qy) = (p+q+r)z.
%  \end{eqnarray*}
%  Dividing both sides by $p+q+r$ gives
%  \[ \frac{px+qy+rz}{p+q+r} = z \]
%  as claimed. 
% \end{solution}

\section{Solitons}\label{sec-solitons}

In this exercise we will investigate the behaviour of some
interesting functions called solitons, which arise in the theory of
water waves in a shallow channel.  Put
\begin{align*}
  q &= \sqrt{2} &
  p &= \log(3 + 2q) \\
  r &= x-4t  &
  s &= q(x-8t) \\
  T &= 32\cosh(2r-p) + 16\cosh(2s-p) + 16 &
  B &= 4(1+q)\cosh(r)\cosh(s) + (4q-8)\exp(r+s) \\
  \phi_0 &= 2\cosh(r)^{-2} &
  \phi_1 &= 4\cosh(s)^{-2} \\
  \phi_2 &= 2\cosh(r-p)^{-2} &
  \phi_3 &= 4\cosh(s-p)^{-2} \\
  \phi_4 &= T/B^2.
\end{align*}
Note that the functions $\phi_i$ depend $x$ (position) and also $t$
(time).  We can therefore differentiate them with respect to $x$ or
with respect to $t$.  In this situation where there are several
variables, it is traditional to use the notation
$\partial\phi_i/\partial x$ and $\partial\phi_i/\partial t$ rather
than $d\phi_i/dx$ and $d\phi_i/dt$.  These can be entered in Maple in
the usual way, as \verb~diff(phi[i],x)~ and \verb~diff(phi[i],t)~. 

Check that the functions $\phi_i$ all satisfy the Korteweg-de Vries
equation: 
\[ \frac{\partial\phi_i}{\partial t} + 
   \frac{\partial^3\phi_i}{\partial x^3} + 
   6 \phi_i \frac{\partial\phi_i}{\partial x} = 0.
\]
Maple will do this quite happily for $i=0,\dotsc,3$.  For $\phi_4$,
however, you need to help by telling Maple to convert all hyperbolic
functions to explicit exponentials at the beginning.  The easiest
thing is to enter \verb~phi[4] := convert(phi[4],exp);~ before working
out the derivatives.

Next, do some plots to investigate the behaviour of these functions.
One approach is to use the \verb~subs()~ command to put $t=1$ (say),
and then plot the result for $-30\leq x\leq 30$.  You can then change
$t$ and do the same again.  It is illuminating to plot several of the
functions $\phi_i$ in the same graph.  

You can also make a movie.  For example:
\begin{verbatim}
with(plots):
animate(
 plot,
 [[phi[2],phi[3]+5,phi[4]+10],x=-30..30],
 t=-3..3,
 frames=100, scaling=constrained, axes=none
);
\end{verbatim}
This makes Maple work quite hard, so it may take a minute or so.  When
the plot appears, click on it; then some playback controls will appear
near the top of the Maple window, which should be fairly
self-explanatory.  Note that we have plotted $\phi_2$, $\phi_3+5$ and
$\phi_4+10$; the $+5$ and $+10$ terms just shift the graphs of
$\phi_3$ and $\phi_4$ upwards, so we can see them more clearly.

Experiment with further movies of this type, including functions like
$\phi_0+\phi_2$ as well as the functions $\phi_i$ themselves.  See if
you can write a clear, self-contained summary of how the functions
$\phi_i$ are related to each other.

Next, the \emph{momentum} of $\phi_i$ is defined to be the integral
$M_i=\int_{-\infty}^\infty \phi_i\,dx$.  As $\phi_i$ depends on $t$ as
well as $x$, you might think that $M_i$ depends on $t$.  However, it
turns out that the dependence on $t$ cancels out, so $M_i$ is a
constant.  Ask Maple to calculate $M_i$ for $i=0,\dotsc,3$, and you
will see that this is true.

Maple is not clever enough to calculate $M_4$ symbolically.  However,
it can do it numerically if we tell it to fix a particular $t$ (say
$t=0$) and give it a hint about the best numerical method to use:
\begin{verbatim}
 M[4] := int(subs(t=0,phi[4]),x=-infinity..infinity,numeric,method=_Gquad);
\end{verbatim}
(If we leave out \verb~method=_Gquad~ then Maple will use a different
method which gives the same answer but takes several minutes, whereas
the method above takes only a few seconds.)  You should then check
that $M_4=M_0+M_3$.

We also define the \emph{energy} of $\phi_i$ to be the integral
$E_i=\int_{-\infty}^\infty\phi_i^2\,dx$.  Repeat the above steps for
the energy, giving exact answers for $E_0,\dotsc,E_3$ and an
approximate answer for $E_4$, and check that $E_4=E_0+E_3$.



\end{document}
